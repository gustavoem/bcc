\documentclass[11pt]{article}

\usepackage{fullpage} 
\usepackage{hyperref}
\usepackage{amsmath}
\usepackage{amssymb}
\usepackage{amsthm}
\usepackage{graphicx}
\usepackage{pgf}
\usepackage{tikz}
\usetikzlibrary{arrows,automata}


\newcommand{\question}[2] {\vspace{0.3in}\noindent{\subsection*{Question #1. #2} \vspace{0.15in}}}

\renewcommand{\part}[1] {{\vspace{0.15in}\noindent\textbf (#1)} \vspace{0.10in}}



%  ----------------------------------------------------------------
%                         Start here
% ----------------------------------------------------------------
 
\begin{document}

\title{Assignment \#1} %Replace X with the appropriate number
\author{\Large Gustavo Estrela de Matos\\ %Replace with your name
CSCE 433: Formal Languages and Automata} %If necessary, replace with your course number and title
\date{\today}
\maketitle


\question{1}{Prove by induction that for $n \ge 1$, $\sum_{i = 1}^{n}\frac{1}{2 ^ i} = 1 - \frac{1}{2^n}$}
\begin{itemize}
\item{Base case:} For the base case we know that $n = 1$, then: \\
    $\sum_{i = 1}^{1}{\frac{1}{2^i}} = {\frac{1}{2}}$; and also $1 - {\frac{1}{2}} = {\frac{1}{2}}$.\\
    Therefore the equation holds for the base case.
\item{Inductive hypothesis:} Assume that the equation holds for $2 \leq n \leq k$ for some integer $k$\\
\item{Inductive step:} For the case where $n = k + 1$, we have that: \\
\begin{equation*}
\begin{aligned}
    \sum_{i = 1}^{k + 1}{\frac{1}{2^i}} & = \sum_{i = 1}^{k}{\frac{1}{2^i}} + {\frac{1}{2^{k + 1}}} \\
                                        & = 1 - \frac{1}{2^k} + \frac{1}{2^{k + 1}} \textup{ (by inductive hypothesis)} \\
                                        & = 1 - (\frac{1}{2^k})(1 - \frac{1}{2}) \\
                                        & = 1 - \frac{1}{2^{k + 1}} \\
\end{aligned}
\end{equation*}
    Then, if the equations holds for $k$ it also holds for $k + 1$. Now by induction, using the base case, inductive hypothesis and inductive step, we have that for all $n \geq 1$ the equation holds, as we wanted to proof.

\end{itemize}


\question{2}{Show that any integer postage greater than 7 cents can be formed by using only 3-cent and 5-cent stamps}
Basically, what we want to proof is that, for any $n > 7$, $n$ can be written as 
\begin{align}
    n = 3a + 5b \label{eq2}
\end{align}
where $a$ and $b$ non negative numbers. Let $k$ be an integer such that $k > 7$ and take $k \mod 3$:
\begin{itemize}
\item{if $k \equiv 0 \mod 3$:} \\
    Then, for some integer $q \geq 0$, $k = 3q$, so equation \ref{eq2} holds.

\item{if $k \equiv 1 \mod 3$:} \\
    We have that
\begin{equation*}
\begin{aligned}
    k - 10 &\equiv 1 - 10 \mod 3\\
    k - 10 &\equiv 0 \mod 3 
\end{aligned}
\end{equation*}
    and, since $k > 7$ and $k \equiv 1 \mod 3$ we have that $k \geq 10$, hence $k - 10 \geq 0$ and there is $q \geq$ integer such that
\begin{equation*}
\begin{aligned}
    k - 10 &= 3q \\
    k      &= 3(q) + 5(2) \\
\end{aligned}
\end{equation*}
    Therefore, equation \ref{eq2} also holds for this case.

\item{if $k \equiv 2 \mod 3$:} \\
    We have that
\begin{equation*}
\begin{aligned}
    k - 5 &\equiv 2 - 5 \mod 3\\
    k - 5 &\equiv 0 \mod 3 
\end{aligned}
\end{equation*}
    and, since $k > 7$, there is $q \geq 0$ integer such that
\begin{equation*}
\begin{aligned}
    k - 5 &= 3q \\
    k &= 3(q) + 5(1) 
\end{aligned}
\end{equation*}
    Therefore, equation \ref{eq2} holds again.
\end{itemize}
Since equation \ref{eq2} holds for all possible

\question{3}{Constructing DFA's}

\end{document}

