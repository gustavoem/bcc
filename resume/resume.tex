%!TEX TS-program = xelatex
%!TEX encoding = UTF-8 Unicode
% Awesome CV LaTeX Template for CV/Resume
%
% This template has been downloaded from:
% https://github.com/posquit0/Awesome-CV
%
% Author:
% Claud D. Park <posquit0.bj@gmail.com>
% http://www.posquit0.com
%
% Template license:
% CC BY-SA 4.0 (https://creativecommons.org/licenses/by-sa/4.0/)
%


%-------------------------------------------------------------------------------
% CONFIGURATIONS
%-------------------------------------------------------------------------------
% A4 paper size by default, use 'letterpaper' for US letter
\documentclass[11pt, a4paper]{awesome-cv-res}

% Configure page margins with geometry
\geometry{left=1.4cm, top=.8cm, right=1.4cm, bottom=1.8cm, footskip=.5cm}

% Specify the location of the included fonts
\fontdir[./fonts/]

% Color for highlights
% Awesome Colors: awesome-emerald, awesome-skyblue, awesome-red, awesome-pink, awesome-orange
%                 awesome-nephritis, awesome-concrete, awesome-darknight
\colorlet{awesome}{awesome-red}
% Uncomment if you would like to specify your own color
% \definecolor{awesome}{HTML}{CA63A8}

% Colors for text
% Uncomment if you would like to specify your own color
% \definecolor{darktext}{HTML}{414141}
% \definecolor{text}{HTML}{333333}
% \definecolor{graytext}{HTML}{5D5D5D}
% \definecolor{lighttext}{HTML}{999999}

% Set false if you don't want to highlight section with awesome color
\setbool{acvSectionColorHighlight}{false}

% If you would like to change the social information separator from a pipe (|) to something else
\renewcommand{\acvHeaderSocialSep}{\quad\textbar\quad}


%-------------------------------------------------------------------------------
%	PERSONAL INFORMATION
%	Comment any of the lines below if they are not required
%-------------------------------------------------------------------------------
\name{Gustavo Estrela de Matos}{}
%\position{Computer Science Student}
\mobile{(+1) 979-402-9614} 
\email{estrela.gustavo.matos@gmail.com}
\github{gustavoem}


%-------------------------------------------------------------------------------
\begin{document}

% Print the header with above personal informations
\makecvheader
% Print the footer with 3 arguments(<left>, <center>, <right>)
% Leave any of these blank if they are not needed
\makecvfooter
  {}
  {}
  {\thepage}


%-------------------------------------------------------------------------------
%	CV/RESUME CONTENT
%	Each section is imported separately, open each file in turn to modify content
%-------------------------------------------------------------------------------

%-------------------------------------------------------------------------------
%     EDUCATION
%-------------------------------------------------------------------------------
\cvsection{Education}
\begin{cventries}
%---------------------------------------------------------
\cventry
{B.S. in Computer Science}
{Texas A\&M University}
{College Station, Texas}
{September 2015 - Present}
{Non-degree seeking exchange student participant of a Government of Brazil program Science Without Borders. \newline GPA: 3.77/4}
\newline 
\newline

\cventry
{B.S. in Computer Science} % Degree
{Institute of Mathematics and Statistics (University of São Paulo)} % Institution
{Sao Paulo, Brazil} % Location
{February 2013 - Present} % Date(s)
{Expected graduation: 2017 \newline GPA: 8.9/10}
\end{cventries}
%---------------------------------------------------------

%-------------------------------------------------------------------------------
%     EXPERIENCES
%-------------------------------------------------------------------------------
\cvsection{Experiences}
\begin{cventries}
%---------------------------------------------------------
\cventry
{Undergraduate Researcher}
{São Paulo Research Foundation and Instituto Butantan}
{São Paulo, Brazil}
{January 2015 - July 2015}
{In this opportunity I developed alternative algorithms to solve the U-Curve problem, an optimization problem defined over a boolean lattice in which the objective function describes u-shaped curves for every chain of the lattice.}
\newline 
\newline

\cventry
{Professor Assitant} 
{Institute of Mathematics and Statistics (University of São Paulo)} 
{Sao Paulo, Brazil}
{August 2014 - December 2014}
{Professor assistent of Introduction to Computer Science class. I graded student homeworks, answered student questions and replaced the professor in the first two lectures of the course.}
\end{cventries}
%---------------------------------------------------------

%-------------------------------------------------------------------------------
%     SKILLS
%-------------------------------------------------------------------------------
\cvsection{Skills}
\begin{cvskills}
\cvskill
{Programming} % Category
{C/C++, JAVA, Python, Octave (Matlab), LaTeX, MySQL, Linux, Arduino, Shell, Git, Vim.} % Skills

%---------------------------------------------------------
\cvskill
{Languages} % Category
{English, Portuguese} % Skills

%---------------------------------------------------------
\end{cvskills}
%---------------------------------------------------------

%-------------------------------------------------------------------------------
%     EXTRA-CURRICULAR ACTIVITIES
%-------------------------------------------------------------------------------
\cvsection{Extra-curricular Activities}
\begin{cventries}

\cventry
{Member}
{Computer Science Class' Representant}
{São Paulo, Brazil}
{February 2013 - August 2015}
{}

\cventry
{Co-founder}
{University of São Paulo Open Source Hardware Student Group}
{São Paulo, Brazil}
{November 2013 - August 2015}
{The group was founded by computer science students in an attempt to foment the studies of open source hardware such as Arduino between students and local community.}
\newline
\newline

\cventry
{Member}
{Texas A\&M Table Tennis Club}
{College Station, Texas}
{January 2016 - Present}
{}

\cventry
{Member}
{Institute of Mathematics and Statistics Baseball and Softball Club}
{São Paulo, Brazil}
{February 2013 - August 2015}
{}




%---------------------------------------------------------
\end{cventries}
%---------------------------------------------------------


\end{document}
