\documentclass[12pt]{article}
\usepackage[a4paper,margin=1in,footskip=0.25in]{geometry} % set margins
\usepackage[portuguese]{babel}
\usepackage[utf8]{inputenc}
\usepackage{hyperref} 
\usepackage{amsmath}
\usepackage{amssymb}
\usepackage{amsthm}
\usepackage{graphicx} % include graphics
\usepackage{indentfirst}


% Macros
\newcommand{\questao}[1] {\vspace{12pt} \noindent \large \textbf{Questão #1:} \normalsize}
\renewcommand{\part}[1] {\noindent\textbf (#1)}
\renewcommand{\familydefault}{\sfdefault} % sans-serif


%  ----------------------------------------------------------------
%                         Start here
% ----------------------------------------------------------------
 
\begin{document}
\title{EP1 - MAC0210}
\begin{flushleft}
    \textbf{\fontsize{20pt}{2em}\selectfont 
        MAC0210 - Exercício Programa 1\\}
    \fontsize{10pt}{1em}\selectfont{Professor: 
        \href{mailto:egbirgin@gmail.com}{Ernesto G. Birgin}\\
    Monitor: 
        \href{mailto:estrela.gustavo.matos@gmail.com}{Gustavo Estrela}\\}
\end{flushleft}


\section {Parte 1: Aritmética de Ponto Flutuante}
    Essa parte do EP consiste em resolver quatro exercícios sobre 
aritmética de ponto flutuante. Para solucionar os problemas é
recomendado a leitura do livro "Numerical Computing with IEEE Floating
Point Arithmetic", presente na 
\href{http://ime.usp.br/~egbirgin/mac210/biblio.html}{bibliografia} do 
curso. Todos os exercícios da parte 1 desse EP foram inspirados em 
exercícios do livro, indicados entre parênteses.

    Você deve subir um arquivo pdf no paca com as suas soluções.
Além disso você deve entregar na aula do dia 20 de setembro uma 
cópia impressa do mesmo pdf que foi entregue no paca.

\questao{1 (3.11)} 
Suponha que temos um sistema de representação de ponto flutuante com
base 2 e,
\begin{center}
    \begin{tabular}{r l}
             & $x = \pm S \times 2^E$,\\ 
        com  & $S = (0.1b_2b_3b_4...b_{24})$, \\
        i.e, & $\frac{1}{2} < S < 1$
    \end{tabular}
\end{center}
onde o expoente $-128 < E < 127$. \\ 
\part{a} Qual é o maior número de ponto flutuante desse sistema? \\
\part{b} Qual é o menor número de ponto flutuante positivo desse 
sistema? \\
\part{c} Qual é o menor inteiro positivo que não é exatamente 
representável nesse sistema? \\

\questao{2 (5.1)}
Qual é a representação do número $1/10$ no formato IEEE single para
cada um dos quatro modos de arredondamento? E para os números 
$1 + 2^{-25}$ e $2^{130}$?\\

\questao{3 (6.4)}
Qual é o maior número de ponto flutuante $x$ tal que $1 \oplus x$ é
exatamente 1, assumindo que o formato usado é IEEE single e modo de 
arredondamento para o mais próximo? E se o formato for IEEE double?

\questao{4 (6.8)} Em aritmética exata, a soma é um operador
comutativo e associativo. O operador de soma de ponto flutuante é 
commutativo? E associativo? Explique.

\section {Parte 2: Método de Newton}
Quando se aplica o Método de Newton a uma função com mais de uma raíz,
temos que a raíz que será obtida pelo método depende do ponto inicial 
escolhido. Nesta parte do EP vamos extender o Método de Newton para um
domínio complexo e estudar suas \textbf{bacias de convergência}, o 
conjunto de pontos iniciais que convergem para uma mesma raíz da função
estudada.

Você deve implementar o Método de Newton em Octave e usar o script 
disponível em 
\end{document}

