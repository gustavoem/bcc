%!TEX TS-program = xelatex
%!TEX encoding = UTF-8 Unicode
% Awesome CV LaTeX Template for CV/Resume
%
% This template has been downloaded from:
% https://github.com/posquit0/Awesome-CV
%
% Author:
% Claud D. Park <posquit0.bj@gmail.com>
% http://www.posquit0.com
%
% Template license:
% CC BY-SA 4.0 (https://creativecommons.org/licenses/by-sa/4.0/)
%

%-------------------------------------------------------------------------------
% CONFIGURATIONS
%-------------------------------------------------------------------------------
% A4 paper size by default, use 'letterpaper' for US letter
\documentclass[11pt, a4paper]{awesome-cv-res}

% Configure page margins with geometry
\geometry{left=1.4cm, top=.8cm, right=1.4cm, bottom=1.8cm, footskip=.5cm}

% Specify the location of the included fonts
\fontdir[./fonts/]

% Color for highlights
% Awesome Colors: awesome-emerald, awesome-skyblue, awesome-red, awesome-pink, awesome-orange
%                 awesome-nephritis, awesome-concrete, awesome-darknight
\colorlet{awesome}{awesome-red}
% Uncomment if you would like to specify your own color
 \definecolor{awesome}{HTML}{14255F}

% Colors for text
% Uncomment if you would like to specify your own color
\definecolor{darktext}{HTML}{414141}
% \definecolor{text}{HTML}{333333}
\definecolor{graytext}{HTML}{5D5D5D}
\definecolor{lighttext}{HTML}{999999}

% Set false if you don't want to highlight section with awesome color
\setbool{acvSectionColorHighlight}{false}

% If you would like to change the social information separator from a pipe (|) to something else
\renewcommand{\acvHeaderSocialSep}{\quad\textbar\quad}


%-------------------------------------------------------------------------------
%	PERSONAL INFORMATION
%	Comment any of the lines below if they are not required
%-------------------------------------------------------------------------------
\name{Gustavo Estrela de Matos}{}
%\address{235 Agostinho dos Santos Street, São Paulo - São Paulo}
%\position{Computer Science Student}
\mobile{(+55) 011 975-934-129}
\email{estrela.gustavo.matos@gmail.com}
\github{gustavoem}


%-------------------------------------------------------------------------------
\begin{document}

% Print the header with above personal informations
\makecvheader
% Print the footer with 3 arguments(<left>, <center>, <right>)
% Leave any of these blank if they are not needed
\makecvfooter
{}
{}
{\thepage}


%-------------------------------------------------------------------------------
%	CV/RESUME CONTENT
%	Each section is imported separately, open each file in turn to modify content
%-------------------------------------------------------------------------------

%-------------------------------------------------------------------------------
%     EDUCATION
%-------------------------------------------------------------------------------
\cvsection{Education}
\begin{cventries}
%---------------------------------------------------------
\cventry
{Master of Science in Computer Science}
{Institute of Mathematics and Statistics (University of São Paulo)} % Institution
{São Paulo, Brazil} % Location
{January 2018 - March 2020}
{Dissertation of title "Identification of cell signaling pathways 
based on biochemical reaction kinetics repositories". Research project 
awarded with a São Paulo Research Foundation (FAPESP) scholarship.}
\newline

\cventry
{Bachelor of Science in Computer Science} % Degree
{} % Institution
{} % Location
{February 2013 - December 2017} % Date(s)
{High Academic Merit Award Ranked \#3 out of 50 students.\newline
GPA: 9/10.}
\newline

\cventry
{Study abroad program in Computer Science}
{Texas A\&M University}
{College Station, Texas}
{September 2015 - May 2016}
{Non-degree seeking exchange student participant of the Government of Brazil program Science Without Borders. \newline GPA: 3.75/4}
\end{cventries}
%---------------------------------------------------------

%-------------------------------------------------------------------------------
%     EXPERIENCES
%-------------------------------------------------------------------------------
\cvsection{Experiences}
\begin{cventries}
%---------------------------------------------------------
\cventry
{Graduate Researcher (with FAPESP Scholarship)}
{Butantan Institute}
{São Paulo, Brazil}
{January 2018 - December 2019}
{\href{https://bv.fapesp.br/en/bolsas/175684/identification-of-cell-signaling-pathways-based-on-biochemical-reaction-kinetics-repositories/}
{\color{awesome}\underline {Masters project}} in a team of 2 (student 
    and advisor) placed in the Center of Toxins, Immune-Response and 
    Cell Signaling (CeTICS) of Butantan Institute. This project has the 
    goal of finding models based on systems of ordinary differential 
    equations for cell signaling networks. We proposed on this project 
    to group data from biological databases such as KEGG and BioModels 
    to systematically find models that can reproduce biological 
    experiments. To measure model quality we used Bayesian methods of 
    likelihood estimation.}

\cventry
{Undergraduate Researcher (with FAPESP Scholarship)}
{}
{}
{May 2017 - December 2017}
{\href{https://bv.fapesp.br/en/bolsas/170553/design-of-poset-forest-based-algorithms-for-the-u-curve-optimization-problem/}
{\color{awesome} \underline{Scientific initiation}} in a team of 2 
    (student and advisor). In this project we created new parallel 
    algorithms to solve the U-Curve problem based on forest search and 
    divide and conquer; some of these algorithms are competitive with 
    state of the art in feature selection. The results were presented 
    as a \href{http://linux.ime.usp.br/~gustavoem/mac0499}
    {\color{awesome} \underline{conclusion project}} for the bachelor 
    title in computer science.}

\cventry
{Undergraduate Researcher (with FAPESP Scholarship)}
{}
{}
{January 2015 - July 2015}
{\href{https://bv.fapesp.br/en/bolsas/156441/studies-of-efficient-data-structures-to-tackle-the-u-curve-optimization-problem/}
{\color{awesome} \underline{Scientific initiation}} in a team of 2 
    (student and advisor). In this opportunity we studied the use of new 
    data structures for the U-Curve problem. As a result we created a 
    new algorithm, UCSR, described on a \href{https://www.sciencedirect.com/science/article/pii/S0020025518306789?via\%3Dihub}
    {published paper}.}


\cventry
{Undergraduate Researcher}
{Texas A\&M University}
{}
{June 2016 - July 2016}
{In a team of 2 (student and advisor), studied forest based algorithms
for the U-Curve problem. Also experimented stochastic versions of state
    of the art algorithms.}
\end{cventries}
%---------------------------------------------------------

%-------------------------------------------------------------------------------
%     PUBLICATIONS
%-------------------------------------------------------------------------------
\cvsection{Publications}
\begin{cventries}
\cventry
    {REIS, MARCELO S.; ESTRELA, GUSTAVO; FERREIRA, CARLOS EDUARDO; BARRERA, JUNIOR.}
    {featsel: A framework for benchmarking of feature selection algorithms and cost functions.}
    {SoftwareX, v. 6, p. 193-197, 2017.}
    {}
    {In the context of machine learning, feature selection is an 
    interesting tool to reduce data complexity. Since the feature 
    selection problem is NP-hard, many algorithms exist to solve the 
    problem and a standard benchmark is desirable. On this paper we 
    describe featsel, a framework implemented in C++ to construct and 
    benchmark different feature selection algorithms and cost 
    functions.}
\cventry
    {REIS, MARCELO S.; ESTRELA, GUSTAVO; FERREIRA, CARLOS EDUARDO; BARRERA, JUNIOR.}
    {Optimal Boolean lattice-based algorithms for the U-curve optimization problem.}
    {Information Sciences, \newline2018.}
    {}
    {The U-curve problem is a particular case of the feature selection
    problem when the cost function describes U-shaped curves in every
    chain of search space. In this paper we present two new algorithms
    and compare them to other state of the art algorithms using the 
    featsel framework.}
\end{cventries}
%---------------------------------------------------------

%-------------------------------------------------------------------------------
%     SKILLS
%-------------------------------------------------------------------------------
\cvsection{Skills}
\begin{cvskills}
\cvskill
{Programming} % Category
{C/C++, Python, Git, Octave (Matlab), Java, Bash, Perl,
Android, React, Arduino, Ruby, JavaScript}

\cvskill
{Languages} % Category
{Portuguese (native), English (advanced; TOEFL IBT score: 93)} % Skills
\end{cvskills}
%---------------------------------------------------------

%-------------------------------------------------------------------------------
%     EXTRA-CURRICULAR ACTIVITIES
%-------------------------------------------------------------------------------
\cvsection{Other\vspace{-1em}}
\begin{cventries}
\cventry
    {Member\vspace{-3em}}
    {\vspace{-3em}Texas A\&M Table Tennis Club}
{}
{January 2016 - May 2016}
{}

\cventry
{Co-founder}
{University of São Paulo Open Source Hardware Student Group}
{}
{November 2013 - August 2015}
{The group was founded by computer science students in an attempt to 
foment the studies of open source hardware such as Arduino between 
students and local community. }

\cventry
{Member}
{Computer Science Class' Representative}
{}
{February 2013 - August 2015}
{Class representatives in our department are channels of communication
    between students and professors. I also participated on the 
    department project of gathering lecture quality feedback from 
    students.}

\cventry
{Member and organizer}
{Institute of Mathematics and Statistics Baseball and Softball Club}
{}
{February 2013 - Present}
{I participate in many practice sessions and championships of the club 
    every year. During the period of 2014 I worked as the club 
    treasurer.}

%---------------------------------------------------------
\end{cventries}
%---------------------------------------------------------
\end{document}
