%!TEX TS-program = xelatex
%!TEX encoding = UTF-8 Unicode
% Awesome CV LaTeX Template for CV/Resume
%
% This template has been downloaded from:
% https://github.com/posquit0/Awesome-CV
%
% Author:
% Claud D. Park <posquit0.bj@gmail.com>
% http://www.posquit0.com
%
% Template license:
% CC BY-SA 4.0 (https://creativecommons.org/licenses/by-sa/4.0/)
%

%-------------------------------------------------------------------------------
% CONFIGURATIONS
%-------------------------------------------------------------------------------
% A4 paper size by default, use 'letterpaper' for US letter
\documentclass[11pt, a4paper]{awesome-cv-res}

% Configure page margins with geometry
\geometry{left=1.4cm, top=.8cm, right=1.4cm, bottom=1.8cm, footskip=.5cm}

% Specify the location of the included fonts
\fontdir[./fonts/]

% Color for highlights
% Awesome Colors: awesome-emerald, awesome-skyblue, awesome-red, awesome-pink, awesome-orange
%                 awesome-nephritis, awesome-concrete, awesome-darknight
\colorlet{awesome}{awesome-red}
% Uncomment if you would like to specify your own color
 \definecolor{awesome}{HTML}{14255F}

% Colors for text
% Uncomment if you would like to specify your own color
\definecolor{darktext}{HTML}{414141}
% \definecolor{text}{HTML}{333333}
\definecolor{graytext}{HTML}{5D5D5D}
\definecolor{lighttext}{HTML}{999999}

% Set false if you don't want to highlight section with awesome color
\setbool{acvSectionColorHighlight}{false}

% If you would like to change the social information separator from a pipe (|) to something else
\renewcommand{\acvHeaderSocialSep}{\quad\textbar\quad}


%-------------------------------------------------------------------------------
%   PERSONAL INFORMATION
%   Comment any of the lines below if they are not required
%-------------------------------------------------------------------------------
\name{Gustavo Estrela de Matos}{}
%\address{235 Agostinho dos Santos Street, São Paulo - São Paulo}
%\position{Computer Science Student}
\mobile{(+55) 011 975-934-129}
\email{estrela.gustavo.matos@gmail.com}
\github{gustavoem}


%-------------------------------------------------------------------------------
\begin{document}

% Print the header with above personal informations
\makecvheader
% Print the footer with 3 arguments(<left>, <center>, <right>)
% Leave any of these blank if they are not needed
\makecvfooter
{}
{}
{\thepage}


%-------------------------------------------------------------------------------
%   CV/RESUME CONTENT
%   Each section is imported separately, open each file in turn to modify content
%-------------------------------------------------------------------------------

%-------------------------------------------------------------------------------
%     EDUCATION
%-------------------------------------------------------------------------------
\cvsection{Educação}
\begin{cventries}
%---------------------------------------------------------
\cventry
{Mestrado em Ciência da Computação}
{Instituto de Matemática e Estatística - Universidade de São Paulo} % Institution
{São Paulo, Brasil} % Location
{Janeiro 2018 - Março 2020}
{Dissertação com título ``Identificação de vias de sinalização celular baseada em repositórios de cinética de reações bioquímicas". Projeto realizado com bolsa FAPESP.}
\newline

\cventry
{Bacharelado em Ciência da Computação} % Degree
{} % Institution
{} % Location
{Fevereiro 2013 - Dezembro 2017} % Date(s)
{Mérito por Desempenho Acadêmico. Ranqueado \#3 de 50 alunos.\newline
média: 9/10.}
\newline

\cventry
{Estudos no Exterior em Ciência da Computação}
{Texas A\&M University}
{College Station, Texas}
{Setembro 2015 - Maio 2016}
{Período de estudos no exterior como bolsista do programa Ciências Sem Fronteiras. \newline média: 3.75/4}
\end{cventries}
%---------------------------------------------------------

%-------------------------------------------------------------------------------
%     EXPERIENCES
%-------------------------------------------------------------------------------
\cvsection{Experiência}
\begin{cventries}
%---------------------------------------------------------
\cventry
{Pesquisador de mestrado (com bolsa FAPESP)}
{Instituto Butantan}
{São Paulo, Brasil}
{Janeiro 2018 - Dezembro 2019}
{\href{https://bv.fapesp.br/en/bolsas/175684/identification-of-cell-signaling-pathways-based-on-biochemical-reaction-kinetics-repositories/}
{\color{awesome}\underline {Projeto de mestrado}} em um time de 2 
    (aluno e orientador) sediado no Centro de Toxinas, Resposta-imune
    e sinalização celular do Instituto Butantan. Este projeto tem como
    objetivo identificar modelos baseados em sistemas de equações 
    diferenciais para redes de sinalização celular. Propomos neste
    projeto agrupar dados de bancos de dados de biologia, como KEGG,
    para sistematicamente encontrar modelos que reproduzam experimentos
    biológicos. Para medir a qualidade dos modelos, utilizamos um 
    método Bayesiano de estimação de verossimilhança.}

\cventry
{Pesquisador de graduação (com bolsa FAPESP)}
{}
{}
{Maio 2017 - Dezembro 2017}
{\href{https://bv.fapesp.br/en/bolsas/170553/design-of-poset-forest-based-algorithms-for-the-u-curve-optimization-problem/}
{\color{awesome} \underline{Iniciação científica}} em um time de 2
    (aluno e professor). Neste projeto, criamos novos algoritmos 
    paralelos para resolver o problema U-Curve em uma abordagem baseada
    em buscas em florestas e divisão e conquista; alguns algoritmos
    criados são competitivos com o estado da arte na área. Os resultados
    foram apresentados como 
    \href{http://linux.ime.usp.br/~gustavoem/mac0499}
    {\color{awesome} \underline{trabalho de conclusão}} para o título
    de bacharel em ciência da computação.}

\cventry
{Pesquisador de graduação (com bolsa FAPESP)}
{}
{}
{Janeiro 2015 - July 2015}
{\href{https://bv.fapesp.br/en/bolsas/156441/studies-of-efficient-data-structures-to-tackle-the-u-curve-optimization-problem/}
{\color{awesome} \underline{Iniciação científica}} em um time de 2 
    (aluno e orientador). Durante este projeto, estudamos o uso de uma
    nova estrutura de dados para algoritmos do problema U-Curve. Como
    resultado, obtemos o algoritmo UCSR, descrito em um \href{https://www.sciencedirect.com/science/article/pii/S0020025518306789?via\%3Dihub}
    {artigo publicado}.
}


\cventry
{Pesquisador de graduação}
{Texas A\&M University}
{}
{June 2016 - July 2016}
{em um time de 2 (aluno e orientador), estudamos o problema U-Curve
    em um contexto estocástico.}
\end{cventries}
%---------------------------------------------------------

%-------------------------------------------------------------------------------
%     PUBLICATIONS
%-------------------------------------------------------------------------------
\cvsection{Publicações}
\begin{cventries}
\cventry
    {REIS, MARCELO S.; ESTRELA, GUSTAVO; FERREIRA, CARLOS EDUARDO; BARRERA, JUNIOR.}
    {featsel: A framework for benchmarking of feature selection algorithms and cost functions.}
    {SoftwareX, v. 6, p. 193-197, 2017.}
    {}
    {Este artigo descreve o arcabouço featsel, implementado em C++
    para permitir a construção e teste de algoritmos de seleção de 
    características.}
\cventry
    {REIS, MARCELO S.; ESTRELA, GUSTAVO; FERREIRA, CARLOS EDUARDO; BARRERA, JUNIOR.}
    {Optimal Boolean lattice-based algorithms for the U-curve optimization problem.}
    {Information Sciences, \newline2018.}
    {}
    {Este artigo apresenta dois novos algoritmos para o problema U-Curve
    e os compara com algoritmos estado da arte, usando o arcabouço
    featsel.}
\end{cventries}
%---------------------------------------------------------

%-------------------------------------------------------------------------------
%     SKILLS
%-------------------------------------------------------------------------------
\cvsection{Habilidades}
\begin{cvskills}
\cvskill
{Programação} % Category
{C/C++, Python, Git, Octave (Matlab), Java, Bash, Perl,
Android, React, Arduino, Ruby, JavaScript}

\cvskill
{Idiomas} % Category
{Português (nativo), Inglês (avançado; nota TOEFL IBT: 93)} % Skills
\end{cvskills}
%---------------------------------------------------------

%-------------------------------------------------------------------------------
%     EXTRA-CURRICULAR ACTIVITIES
%-------------------------------------------------------------------------------
\cvsection{Outros\vspace{-1em}}
\begin{cventries}
\cventry
    {Membro\vspace{-3em}}
    {\vspace{-3em}Texas A\&M Table Tennis Club}
{}
{Janeiro 2016 - Maio 2016}
{}

\cventry
{Co-fundador}
{Hardware Livre USP}
{}
{Novembro 2013 - Agosto 2015}
{O grupo Hardware Livre USP foi criado por alunos de ciência da 
    computação com objetivo de fomentar o estudo de hardware livre,
    como, por exemplo, o Arduino.
}

\cventry
{Membro}
{Representação de Classes}
{}
{Fevereiro 2013 - Agosto 2015}
{Representação da turma de ingresso 2013 no curso de ciência da 
    computação. Os representantes eram responsáveis por ajudar a 
    comunicação entre docentes e alunos, e também por ajudar no
    processo de avaliação das disciplinas.}

\cventry
{Membro e organizador}
{IME-USP Beisebol e Softbol Clube}
{}
{Fevereiro 2013 - Present}
{Iniciei a prática do beisebol e softbol na graduação e continuo
praticando ambos esportes, nas equipes do IME-USP. No ano de 2014
fui tesoureiro do clube.}

%---------------------------------------------------------
\end{cventries}
%---------------------------------------------------------
\end{document}
