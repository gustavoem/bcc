\documentclass[11pt]{article}

\usepackage{fullpage} 
\usepackage{hyperref}
\usepackage{amsmath}
\usepackage{amssymb}
\usepackage{amsthm}
\usepackage{graphicx}
\usepackage{pgf}
\usepackage{tikz}
\usetikzlibrary{arrows,automata}
\usepackage{indentfirst}

\newcommand{\question}[2] {\vspace{0.3in}\noindent{\subsection*{Question #1. #2} \vspace{0.15in}}}

\renewcommand{\part}[1] {{\vspace{0.15in}\noindent\textbf (#1)} \vspace{0.10in}}



%  ----------------------------------------------------------------
%                         Start here
% ----------------------------------------------------------------
 
\begin{document}

\title{Assignment \#4} %Replace X with the appropriate number
\author{\Large Gustavo Estrela de Matos\\ %Replace with your name
CSCE 433: Formal Languages and Automata} %If necessary, replace with your course number and title
\date{\today} 

\maketitle

\question{1}{}




\question{2}{}

\part{a} 
$
\begin{cases}
q_0 = \epsilon q_1 + bq_2 \\
q_1 = a q_1 + \epsilon q_2 + \epsilon \\
q_2 = \epsilon
\end{cases}
$

\begin{itemize}
    \item{Substitute $q_2$ into the second equation}
\end{itemize}

\begin{flushleft}
\begin{equation*}
\begin{aligned}
    q_1 &= a q_1 + \epsilon + \epsilon \\
        &= a^* \textup{(by Arden's Lemma)}
\end{aligned}
\end{equation*}
\end{flushleft}




\question{3}{}

\part{a} $G = (V, \Sigma, R, S)$ where:
\begin{itemize}
    \item{$V = \{S\}$}
    \item{$\Sigma = \{a, b\}$}
\end{itemize}
    \par Note that $\epsilon \in L_1$, therefore our first rule is $S \rightarrow \epsilon$. To build other strings we have to add $1$'s into the end and twice this number of $0$'s at the beggining of the string, then the second rule is $S \rightarrow 00S1$.
\begin{itemize}
    \item {$R = \{S \rightarrow \epsilon$, $S \rightarrow 00S1\}$}
\end{itemize}


\part{b} $G = (V, \Sigma, R, S)$ where:

\begin{itemize}
    \item{$V = \{S\}$}
    \item{$\Sigma = \{a, b\}$}
\end{itemize}

\par Note that $\epsilon \in L_1$, therefore our first rule is $S \rightarrow \epsilon$. To build other strings of this language we have to guarantee that $m \geq n$ and keep $m$ and $n$ both odd or even. Which means that:

\par 1 $S \rightarrow aSb$; whenever adding an $a$ you have to add a new $b$ to keep $m \geq n$ and $m - n$ even.
\par 2 $S \rightarrow Sbb$; whenever adding a $b$ we need to add two of them to keep $m - n$ even.
\par Therefore the rules for this language are:

\begin{itemize}    
    \item{$R = \{S \rightarrow \epsilon$, $S \rightarrow Sbb$, $S \rightarrow aSb\}$}
\end{itemize}


\part{c} $G = (V, \Sigma, R, S)$ where:

\begin{itemize}
    \item{$V = \{S, S_1, R_1, T_1, S_2, R_2, T_2\}$}
    \item{$\Sigma = \{a, b, c, d\}$}
\end{itemize}
    
\par Note that the equation $m + n = p + q$ means that for every $a$ or $b$ added we need to add a new $c$ or $d$. To do that we are going to build the string from the outside to inside adding for every $a$ or $b$ a new $d$ or a new $c$. You should also note that it's hard have the pair of rules "for every $a$ add a new $c$" and "for every $b$ add a new $d$" because they intersect each other; to deal with this we are going to create rules for two different situations: $m \geq q$ and $m \leq q$.
\par In the case where $m \geq q$ we are looking for the language $L_1 = \{a^mb^nc^pd^q$ $|$ $m + n = p + q$, $m \geq q\} $ generate the strings with the following rules:
\par 1 $S_1 \rightarrow aS_1d | R_1$
\par 2 $R_1 \rightarrow aR_1c | T_1$
\par 3 $T_1 \rightarrow bT_1c | \epsilon$
\par We do not need a rule that adds a new $d$ when adding a $b$ because we assumed that $m \geq q$ therefore every $d$ of a string would already have been added by rule 1.
\par In the case where $m \leq q$ we are looking for the language $L_2 = \{a^mb^nc^pd^q$ $|$ $m + n = p + q$, $m \leq q\} $ generate the strings with the following rules:
\par 4 $S_2 \rightarrow aS_2d | R_2$
\par 5 $R_2 \rightarrow bR_2d | T_2$
\par 6 $T_2 \rightarrow bT_2c | \epsilon$
\par Simmilarly we do not need a rule that adds a new $c$ when adding a $a$ because we assumed that $m \leq q$ therefore every $a$ of a string would already have been added by rule 4.
    Now the last rule we need is the one that makes the union of $L_1$ and $L_2$: 
\par 7 $S \rightarrow S_1 | S_2$

\begin{itemize}    
    \item{$R = \{S \rightarrow S_1 | S_2$, 
        $S_1 \rightarrow aS_1d | R_1$,
        $R_1 \rightarrow aR_1c | T_1$,
        $T_1 \rightarrow bT_1c | \epsilon$,
        $S_2 \rightarrow aS_2d | R_2$,
        $R_2 \rightarrow bR_2d | T_2$,
        $T_2 \rightarrow bT_2c | \epsilon\}$
      }
\end{itemize}




\question{4}{}
$G = (V, \Sigma, R, S)$ where:
\begin{itemize}
    \item{$V = \{S, R\}$}
    \item{$\Sigma = \{a, b, *, (, ), +, \varepsilon \}$}; to avoid ambiguity we are going to use the symbol $\varepsilon$ to represent the empty string in the regular expressions we want to build.
\end{itemize}

\par To create all the regular expressions we are going to use two different variables: $R$ and $S$. The variable $S$ represents any non-empty regular expression while $R$ can be either $S$ or the empty string $\epsilon$.

\par 1 $S \rightarrow aR$
\par 2 $S \rightarrow bR$
\par 3 $S \rightarrow \varepsilon R$
\par 4 $S \rightarrow S^*R$
\par 5 $S \rightarrow (S)R$
\par 6 $S \rightarrow S + S$
\par 7 $R \rightarrow S$
\par 8 $R \rightarrow \epsilon$
\begin{itemize}    
    \item{$R = \{
        S \rightarrow aR, 
        S \rightarrow bR, 
        S \rightarrow \varepsilon R,
        S \rightarrow S^*R, 
        S \rightarrow (S)R,
        S \rightarrow S + S,
        R \rightarrow S,
        R \rightarrow \epsilon
    \}$
    }
\end{itemize}

\end{document}
