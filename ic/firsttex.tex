\documentclass[12pt]{article}
\usepackage[portuguese]{babel}
\usepackage[utf8]{inputenc}
\usepackage[usenames,dvipsnames]{color}
\usepackage{setspace}
\usepackage{amsmath}
\usepackage{amsfonts}
\usepackage{amssymb}
\usepackage{mathtools}
\usepackage[hmargin=2cm,vmargin=1.5cm,bmargin=3cm]{geometry}
\usepackage{tikz}
\usepackage{textcomp}
\definecolor{myblue}{RGB}{80,80,160}
\definecolor{mygreen}{RGB}{80,160,80}

\begin{document}
\title{Estudo de estruturas de dados eficientes para o problema de otimização U-curve} 
\author{Gustavo Estrela de Matos \\ estrela.gustavo.matos@usp.br} 
\date{outubro de 2014}

\maketitle

\def \negritovi {\textbf}

\tableofcontents 
\pagebreak

\section{Introdução}
\subsection{O problema U-Curve}
Na área de reconhemento de padrões, o problema de seleção de características consiste em escolher um subconjunto ótimo de um conjunto de características para algum critério. Esse critério pode ser definido por uma função avaliadora \begin{math}c\end{math} que associa a cada subconjunto de características o seu custo. Supondo a escolha de tal função \begin{math}c\end{math}, podemos reduzir a seleção de características a um problema de busca em que, dado um conjunto finito de características \begin{math}S\end{math}, precisamos achar um subconjunto \begin{math}X\end{math} que minimiza uma função \begin{math}c\end{math}, definida de $\mathcal{P}$\begin{math}(S)\end{math}, o conjunto potência de \begin{math}S\end{math},  para $\mathbb{R_+}$.

Muitas dessas funções custos, que dependem da estimação de uma distribuição de probabilidade conjunta, descrevem uma "curva em U'' nas cadeias maximais do reticulado booleano. Esse é um fenômeno que ocorre devido a Maldição da Dimensionalidade, que é o nome dado aos fenômenos que ocorrem quando analizamos dados em muitas dimensões. No contexto de reconhecimento de padrões, o classificador melhora com o aumento de características consideradas, porém, depois de um certo ponto, o acréscimo de características torna necessário maior o número de amostras, causando maior erro de estimação e maior custo.

O caso particular de seleção de características em que podemos ver o problema como a minimização de uma função custo cujas cadeias descrevem curvas em u é chamado "problema U-curve". Algoritmos que exploram esse problema como o algoritmo U-curve\cite{u-curve algorithm} podem ser utilizados para solucionar problemas de seleção de características como o projeto de W-operadores. Além do algoritmo "U-curve", outras soluções foram propostas para o "problema U-curve". Dentre eles, algoritmos \emph{branch-and-bound} como U-Curve-Branch-and-Bound (UBB), Poset-Forest-Search (PFS)\cite{msreis thesis} e o iUBB\cite{iubb}.

\subsection{O algoritmo U-Curve-Search}
O algoritmo U-curve mostrou-se não ser ótimo, e sim, sub-ótimo, isto é, o algoritmo pode não encontrar o subconjunto de características de custo mínimo\cite{msreis thesis, ucs paper}. Essa falha levou a elaboração do algoritmo U-Curve-Search (UCS), que é ótimo e se mostra muito promissor, pois em experimentos com instâncias artificiais e com dados práticos o UCS computou a função custo poucas vezes comparado aos outros algoritmos\cite{msreis thesis, ucs paper, math morph paper}.

O U-Curve-Search faz a procura do subconjunto de custo mínimo com o auxílio de coleções de restrições inferiores e superiores do espaço de busca. Por conta disso, as rotinas que fazem consultas e atualizações das restrições inferiores e superiores são chamadas várias vezes durante a execução do algoritmo. Essas coleções de restrições foram implementados usando listas duplamente ligadas e, portanto, a busca por um elemento nessas coleções é \begin{math}O(n|\end{math}$\mathcal{R}$\begin{math}|)\end{math}, no qual \begin{math}n\end{math} é a cardinalidade do conjunto \begin{math}S\end{math} e $\mathcal{R}$ é a coleção de restrições inferior ou superior.



\section{Objetivos}

\section{Plano de estudos}

\section{Materiais e métodos}

\section{Formas de análise dos estudos}

\begin{thebibliography}{9}

\bibitem{u-curve algorithm}
Ris, Marcelo, Junior Barrera, and David C. Martins Jr. \emph{U-curve: A branch-and-bound optimization algorithm for U-shaped cost functions on Boolean lattices applied to the feature selection problem}. Pattern Recognition, 43.3 (2010): 557-568.

\bibitem{msreis thesis}
Reis, Marcelo S. "Minimization of decomposable in U-shaped curves functions defined on poset chains–algorithms and applications." PhD thesis, Institute of Mathematics and Statistics, University of São Paulo, Brazil, (2012).
\bibitem{iubb}

Atashpaz-Gargari, Esmaeil, Ulisses M. Braga-Neto, and Edward R. Dougherty. "Improved branch-and-bound algorithm for U-curve optimization." 2013 IEEE International Workshop on Genomic Signal Processing and Statistics (GENSIPS), (2013).

\bibitem{ucs paper}
Reis, Marcelo S., Carlos E. Ferreira, and Junior Barrera. "The U-curve optimization problem: improvements on the original algorithm and time complexity analysis." arXiv preprint arXiv:1407.6067, (2014). 

\bibitem{math morph paper}
Reis, Marcelo S., and Junior Barrera. "Solving Problems in Mathematical Morphology through Reductions to the U-Curve Problem." Mathematical Morphology and Its Applications to Signal and Image Processing. Springer Berlin Heidelberg, (2013): 49-60.

\end{thebibliography}

\end{document}



