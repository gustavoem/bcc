\documentclass[12pt]{article}

\usepackage[portuguese]{babel}
\usepackage[utf8]{inputenc}
\usepackage[usenames,dvipsnames]{color}
\usepackage{setspace}
\usepackage{amsmath}
\usepackage{amsfonts}
\usepackage{amssymb}
\usepackage{mathtools}
\usepackage[hmargin=2cm,vmargin=1.5cm,bmargin=3cm]{geometry}
\usepackage{tikz}
\usepackage{textcomp}
\definecolor{myblue}{RGB}{80,80,160}
\definecolor{mygreen}{RGB}{80,160,80}

\begin{document} % Aqui começa o documento
\title{Estudo de estruturas de dados eficientes para o problema de otimização U-curve} % título
\author{Gustavo Estrela de Matos \\ estrela.gustavo.matos@usp.br} % quem escreveu
\date{outubro de 2014} %data

\maketitle %cria o título

\def \negritovi {\textbf} %Criando comandos

\tableofcontents %índice
\pagebreak % Quebra de página

\section{Introdu\c{c}\~ao} %Cria uma seção

%\\Outros exemplos de caracteres especiais são \# \$ \%  \& \_ \{ \} que podem ser escritos com um ' $\backslash$ ' na frente.
Na área de reconhemento de padrões, o problema de seleção de características trata de escolher um subconjunto ótimo de um conjunto segundo uma função \begin{math}c\end{math}. Mais formalmente, dado um conjunto finito de características 
\begin{math}S\end{math}, precisamos achar um subconjunto \begin{math}X\end{math} que minimiza uma função \begin{math}c\end{math}, definida de $\mathcal{P}$\begin{math}(S)\end{math} para $\mathbb{R_+}$.

\section{Objetivos}

\section{Plano de estudos}

\section{Materiais e m\'etodos}

\section{Materiais e m\'etodos}

\section{Formas de an\'alise dos estudos}

\end{document} %fim do documento. O que vem depois daqui não é gerado

\section{Escondida}
Essa seção não vai aparecer.