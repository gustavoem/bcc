%!TEX TS-program = xelatex
%!TEX encoding = UTF-8 Unicode
% Awesome CV LaTeX Template for CV/Resume
%
% This template has been downloaded from:
% https://github.com/posquit0/Awesome-CV
%
% Author:
% Claud D. Park <posquit0.bj@gmail.com>
% http://www.posquit0.com
%
% Template license:
% CC BY-SA 4.0 (https://creativecommons.org/licenses/by-sa/4.0/)
%


%-------------------------------------------------------------------------------
% CONFIGURATIONS
%-------------------------------------------------------------------------------
% A4 paper size by default, use 'letterpaper' for US letter
\documentclass[11pt, a4paper]{awesome-cv-res}

% Configure page margins with geometry
\geometry{left=1.4cm, top=.8cm, right=1.4cm, bottom=1.8cm, footskip=.5cm}

% Specify the location of the included fonts
\fontdir[./fonts/]

% Color for highlights
% Awesome Colors: awesome-emerald, awesome-skyblue, awesome-red, awesome-pink, awesome-orange
%                 awesome-nephritis, awesome-concrete, awesome-darknight
\colorlet{awesome}{awesome-red}
% Uncomment if you would like to specify your own color
% \definecolor{awesome}{HTML}{CA63A8}

% Colors for text
% Uncomment if you would like to specify your own color
% \definecolor{darktext}{HTML}{414141}
% \definecolor{text}{HTML}{333333}
% \definecolor{graytext}{HTML}{5D5D5D}
% \definecolor{lighttext}{HTML}{999999}

% Set false if you don't want to highlight section with awesome color
\setbool{acvSectionColorHighlight}{false}

% If you would like to change the social information separator from a pipe (|) to something else
\renewcommand{\acvHeaderSocialSep}{\quad\textbar\quad}


%-------------------------------------------------------------------------------
%	PERSONAL INFORMATION
%	Comment any of the lines below if they are not required
%-------------------------------------------------------------------------------
\name{Gustavo Estrela de Matos}{}
%\address{1100 Hensel Drive, Apt. N302, College Station - Texas}
%\position{Computer Science Student}
\mobile{(+55) 11 975-934-129}
\email{estrela.gustavo.matos@gmail.com}
\github{gustavoem}


%-------------------------------------------------------------------------------
\begin{document}

% Print the header with above personal informations
\makecvheader
% Print the footer with 3 arguments(<left>, <center>, <right>)
% Leave any of these blank if they are not needed
\makecvfooter
  {}
  {}
  {\thepage}


%-------------------------------------------------------------------------------
%	CV/RESUME CONTENT
%	Each section is imported separately, open each file in turn to modify content
%-------------------------------------------------------------------------------

%-------------------------------------------------------------------------------
%     EDUCATION
%-------------------------------------------------------------------------------
\cvsection{Educação}
\begin{cventries}
%---------------------------------------------------------
\cventry
{Bacharelado em Ciência da Computação}
{Texas A\&M University}
{College Station, Texas}
{Setembro 2015 - Maio 2016}
{Participante do programa Ciência Sem Fronteiras. \newline Média: 3.6/4}
\newline 
\newline

\cventry
{Bacharelado em Ciência da Computação} % Degree
{Instituto de Matematica e Estatística (Universidade de São Paulo)} % Institution
{São Paulo, Brasil} % Location
{Fevereiro 2013 - Presente} % Date(s)
{Previsão de conclusão: dezembro de 2017 \newline Média: 8.9/10}
\end{cventries}
%---------------------------------------------------------

%-------------------------------------------------------------------------------
%     EXPERIENCES
%-------------------------------------------------------------------------------
\cvsection{Experiências}
\begin{cventries}
%---------------------------------------------------------
\cventry
{Aluno de Iniciação Científica}
{Fundação de Amparo a Pesquisa do Estado de São Paulo e Instituto Butantan}
{São Paulo, Brasil}
{Maio 2017 - Presente}
{Estudos de algoritmos baseados em florestas para o problema U-Curve.}
\newline 
\newline

\cventry
{Monitor de Graduação}
{Instituto de Matematica e Estatistica (Universidade de São Paulo)} 
{São Paulo, Brasil}
{Agosto 2016 - Dezembro 2016}
{Assistente de professor no curso de Laboratório de Métodos Numéricos. As atividades consistiam em avaliar trabalhos, e sanar dúvidas de alunos.}
\newline 
\newline

\cventry
{Aluno de Iniciação Científica}
{Texas A\&M University}
{College Station, Texas}
{Maio 2016 - Julho 2016}
{Estudos de abordagens estocásticas para o problema U-Curve.}
\newline 
\newline

\cventry
{Aluno de Iniciação Científica}
{Fundação de Amparo a Pesquisa do Estado de São Paulo e Instituto Butantan}
{São Paulo, Brasil}
{Janeiro 2015 - Julho 2015}
{Nessa oportunidade trabalhamos com a elaboração de um algoritmo para solucionar o problema U-Curve, um problema de otimização definido sobre um reticulado booleano em que a função objetivo descreve curvas em formato de "u" para todas as cadeias do reticulado.}
\newline 
\newline

\cventry
{Monitor de Graduação} 
{Instituto de Matematica e Estatistica (Universidade de São Paulo)} 
{São Paulo, Brasil}
{Agosto 2014 - Dezembro 2014}
{Assistente de professor no curso de Introdução a Computação. As atividades consistiam em avaliar trabalhos, sanar dúvidas de alunos e substituir o professor nas aulas iniciais do curso.}
\end{cventries}
%---------------------------------------------------------

%-------------------------------------------------------------------------------
%     SKILLS
%-------------------------------------------------------------------------------
\cvsection{Habilidades}
\begin{cvskills}
\cvskill
{Computação} % Category
{C/C++, JAVA, Python, Perl, Octave (Matlab), LaTeX, MySQL, Linux, 
Arduino, Shell, Git, Vim, Android.} % Skills
%---------------------------------------------------------
\cvskill
{Idiomas} % Category
{Português, Inglês.} % Skills

%---------------------------------------------------------
\end{cvskills}
%---------------------------------------------------------

%-------------------------------------------------------------------------------
%     EXTRA-CURRICULAR ACTIVITIES
%-------------------------------------------------------------------------------
\cvsection{Atividades Extra-curriculares}
\begin{cventries}

\cventry
{Membro}
{Texas A\&M Table Tennis Club}
{College Station, Texas}
{Janeiro 2016 - Maio 2016}
{}

\cventry
{Co-fundador}
{Grupo Hardware Livre da Universidade de São Paulo}
{São Paulo, Brasil}
{Novembro 2013 - Agosto 2015}
{O grupo foi fundado por alunos de ciência da computação como uma tentativa de fomentar o estudo de hardware livre, principalmente da placa Arduino, entre os alunos e comunidade local.}
\newline
\newline

\cventry
{Membro}
{Representantes de Classe de Ciência da Computação}
{São Paulo, Brasil}
{Fevereiro 2013 - Agosto 2015}
{}

\cventry
{Membro}
{Clube de Baseball e Softball do IME-USP}
{São Paulo, Brasil}
{Fevereiro 2013 - Agosto 2015}
{}

%---------------------------------------------------------
\end{cventries}
%---------------------------------------------------------
\end{document}
