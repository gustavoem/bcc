\documentclass[12pt]{article}
\usepackage[portuguese]{babel}
\usepackage[utf8]{inputenc}
\usepackage[usenames,dvipsnames]{color}
\usepackage{setspace}
\usepackage{amsmath}
\usepackage{amsfonts}
\usepackage{amssymb}
\usepackage{mathtools}
\usepackage[top=3cm, bottom=2cm, left=3cm, right=2cm]{geometry}
\usepackage{tikz}
\usepackage{textcomp}

% packages added by Marcelo
%
\usepackage{lscape}    % for landscape pages
\usepackage{hyperref}  % to allow hyperlinks
\usepackage{booktabs}  % nicer table borders
\usepackage{subfigure} % add subfigures

\graphicspath{{./figures/}} 

\definecolor{myblue}{RGB}{80,80,160}
\definecolor{mygreen}{RGB}{80,160,80}
\setstretch{1.5}

\begin{document}

% FAPESP demands the usage of double spacing
%
\doublespacing


\begin{center}

  {\LARGE Estudos de estruturas de dados eficientes para abordar o\\
    \bigskip
    problema de otimização U-curve}

  \bigskip
        
  {\large {\bf Aluno:} \href{mailto:gustavo.estrela.matos@usp.br}{Gustavo Estrela de Matos}\\ 
  {\bf Orientador:} \href{mailto:marcelo.reis@butantan.gov.br}{Marcelo da Silva Reis}\\
  \bigskip
  7 de novembro de 2014\\
  }

  \bigskip
  \bigskip

  {\bf Resumo}

\end{center}
O problema de otimização U-curve pode ser utilizado para modelar problemas em diversas áreas; por exemplo, o problema de seleção de características em Reconhecimento de Padrões. Um algoritmo ótimo para abordar o problema U-curve é o U-Curve-Search ({\tt UCS}). Esse algoritmo alcançou resultados promissores na solução desse problema, pois computa poucas vezes a função custo; porém, em sua atual implementação, {\tt UCS} tem problemas de escalabilidade, o que se deve em grande medida à utilização de listas duplamente encadeadas para armazenar o controle do espaço de busca já percorrido pelo algoritmo. Dessa forma, propomos neste projeto de Iniciação Científica a investigação do uso de diagramas de decisão binária reduzidos e ordenados (ROBDDs) como solução de estrutura de dados para controlar o espaço de busca já percorrido pelo algoritmo {\tt UCS}. Utilizaremos ROBDDs para desenvolver uma nova versão do {\tt UCS}, que será implementada e testada utilizando o arcabouço featsel. Realizaremos testes com instâncias artificiais e também dados de problemas reais, tais como o desenho de W-operadores. Esperamos que a nova versão do algoritmo {\tt UCS} seja eficiente também do ponto de vista de consumo de tempo computacional, tornando o algoritmo competitivo para resolver problemas práticos que possam ser descritos como um problema U-curve.

\newpage

\begin{center}

  {\LARGE Studies of efficient data structures to tackle the\\
    \bigskip
    U-curve optimization problem}

  \bigskip
        
  {\large {\bf Student:} \href{mailto:gustavo.estrela.matos@usp.br}{Gustavo Estrela de Matos}\\ 
  {\bf Supervisor:} \href{mailto:marcelo.reis@butantan.gov.br}{Marcelo da Silva Reis}\\
  \bigskip
  November 7, 2014\\
  }

  \bigskip
  \bigskip

  {\bf Abstract}

\end{center}
The U-curve optimization problem may be used to model problems in several fields; for instance, the feature selection problem in Pattern Recognition. An optimal algorithm to tackle the U-curve problem is the U-Curve-Search ({\tt UCS}) algorithm. The usage of {\tt UCS} to solve this problem is promissing, since it computes fewer times the cost function than other algorithms. However, the current implementation of {\tt UCS} has scalability issues, which is mostly due the usage of doubly linked lists to store the control of the search space that was already explored by the algorithm execution. Therefore, in this project, we propose the investigation of the usage of Reduced Ordered Binary Decision Diagrams (ROBDDs) as data structure to control the search space during an execution of the {\tt UCS} algorithm. We intend to use ROBDDs to develop a new version of {\tt UCS}, which will be implemented and tested using the featsel framework. We will carry out tests with artificial instances and also data from real-world problems such as the design of W-operators. We expect that the new version of the {\tt UCS} algorithm will be also efficient from the required computational time point of view, hence making this algorithm competitive to solve practical problems that can be described as instances of the U-curve problem.

\newpage

\tableofcontents 

\newpage

\section{Introdução}
\subsection{O problema U-Curve}
Na área de Reconhecimento de Padrões, o problema de seleção de características consiste em, dado um critério, escolher um subconjunto ótimo de um conjunto de características. Esse critério pode ser definido por uma função avaliadora \begin{math}c\end{math} que associa a cada subconjunto de características o seu custo. Supondo a escolha de tal função \begin{math}c\end{math}, podemos reduzir a seleção de características a um problema de busca em que, dado um conjunto finito de características \begin{math}S\end{math}, precisamos achar um subconjunto \begin{math}X\end{math} que minimiza uma função custo \begin{math}c\end{math}, definida de $\mathcal{P}$\begin{math}(S)\end{math}, o conjunto potência de \begin{math}S\end{math},  para $\mathbb{R_+}$.

Funções custo que dependem da estimação de uma distribuição de probabilidade conjunta descrevem uma ``curva em U" nas cadeias maximais do reticulado Booleano $(\mathcal{P}(S),\subseteq)$. Esse é um fenômeno que ocorre devido à \href{http://en.wikipedia.org/wiki/Curse\_of\_dimensionality}{Maldição da Dimensionalidade}, que por sua vez é o nome dado ao que ocorre quando analisamos dados em muitas dimensões. No contexto de Reconhecimento de Padrões, isso significa que um classificador melhora (tem um menor custo) com o aumento de características consideradas em seu desenho, porém somente até o ponto em que a limitação no número de amostras leve a um maior erro de estimação; neste caso, o classificador piora (tem um maior custo). 

O caso particular de seleção de características em que podemos ver o problema como a minimização de uma função custo cujas cadeias descrevem curvas em U é chamado ``problema U-curve". Um exemplo de instância desse problema é ilustrado na figura~\ref{fig:U-curve}.
Algoritmos que exploram o problema U-curve, tais como o algoritmo {\tt U-Curve}~\cite{u-curve algorithm}, podem ser utilizados para solucionar problemas de seleção de características: por exemplo, em projetos de W-operadores~\cite{u-curve algorithm}. Além do algoritmo {\tt U-Curve}, outras soluções foram propostas para resolver o problema U-curve; dentre elas, algoritmos do tipo \emph{branch-and-bound}, tais como U-Curve-Branch-and-Bound ({\tt UBB}), Poset-Forest-Search ({\tt PFS})~\cite{msreis thesis} e improved-UBB ({\tt iUBB})~\cite{iubb}.


\begin{figure}[!ht]
  \centering 
  \begin{tabular}{c c}
    \subfigure[] {\scalebox{0.75}{\includegraphics[trim=2cm 17.5cm 12.2cm 2cm, clip=true]{Boolean_lattice_3_A.pdf}}  \label{fig:U-curve-example:A}}  
  &
    \subfigure[] {\scalebox{0.30}{\includegraphics[clip=true]{exemplo1.pdf}} \label{fig:U-curve-example:B} }
  \end{tabular}
  \caption{um exemplo de instância do problema U-curve. Figura~\ref{fig:U-curve-example:A}: o diagrama de Hasse de um reticulado Booleano de grau $3$ -- as cadeias do reticulado, cujos custos de seus elementos são definidos através dos números ao lado dos nós, descrevem curvas em U; a cadeia maximal $\{ \emptyset, c, bc, abc \}$ está destacada em negrito. O elemento $bc$, destacado em verde, é mínimo na cadeia (e também no reticulado Booleano). Figura~\ref{fig:U-curve-example:B}: o gráfico dos custos em função dos elementos da cadeia maximal destacada em negrito, mostrando sua curva em U. Figura extraída de Reis~\cite{msreis thesis}.} 
  \label{fig:U-curve} 
\end{figure}

\subsection{O algoritmo U-Curve-Search ({\tt UCS})}
Recentemente foi demonstrado que o algoritmo {\tt U-Curve} é na verdade sub-ótimo, isto é, que o mesmo pode não encontrar um subconjunto de características de custo mínimo~\cite{msreis thesis, ucs paper}. Essa falha levou à elaboração do algoritmo U-Curve-Search ({\tt UCS}), que é ótimo e se mostra muito promissor, pois em experimentos com instâncias artificiais e com dados práticos  UCS computou a função custo poucas vezes quanto comparado a outros algoritmos~\cite{msreis thesis, ucs paper, math morph paper}.

O algoritmo {\tt UCS} faz a procura do subconjunto de custo mínimo com o auxílio de duas coleções de elementos do reticulado Booleano, chamadas de restrições inferiores ($\mathcal{R}_L$) e superiores ($\mathcal{R}_U$). Cada elemento $M$ de $\mathcal{R}_L$ (de modo dual, de $\mathcal{R}_U$) define um intervalo $[\emptyset, M]$ ($[M,S]$) que foi removido do espaço de busca. Dessa forma, o espaço de busca em uma iteração corrente do algoritmo {\tt UCS} é constituído do reticulado Booleano menos os intervalos definidos pelos elementos das coleções de restrições; este fato é ilustrado na figura~\ref{fig:lower_restriction}.

\begin{figure}[!ht]
  \centering 
  \scalebox{0.75}{\includegraphics[trim=2cm 17.5cm 6.2cm 2cm, clip=true]{Boolean_lattice_3.pdf}}  
  \caption{exemplo de definição do espaço de busca corrente utilizando coleções de restrições. Neste caso, temos $\mathcal{R}_L = \{ \{a\}, \{ b,c \}\}$ e $\mathcal{R_U} = \emptyset$, o que implica que os intervalos $[\emptyset, \{a\}]$ e $[\emptyset, \{b,c\}]$ não pertencem ao espaço de busca corrente.} 
  \label{fig:lower_restriction} 
\end{figure}


Como para percorrer o espaço de busca corrente é preciso consultar se um dado elemento pertence ou não ao espaço de busca, as rotinas que fazem consultas e atualizações das coleções de restrições inferiores e superiores são chamadas várias vezes durante a execução do algoritmo. Essas coleções de restrições foram implementadas utilizando listas duplamente encadeadas, o que implica que a busca por um elemento nessas coleções consome \begin{math}O(n|\end{math}$\mathcal{R}_L$\begin{math}|)\end{math} (\begin{math}O(n|\end{math}$\mathcal{R}_U$\begin{math}|)\end{math}) unidades de tempo, no qual \begin{math}n\end{math} é a cardinalidade do conjunto \begin{math}S\end{math} e $\mathcal{R}_L$ ($\mathcal{R}_U$) é a coleção de restrições inferiores (superiores)~\cite{msreis thesis}. A utilização dessa estrutura de dados faz com que o algoritmo {\tt UCS} gaste muito tempo computacional verificando se um dado elemento é coberto por algum dos intervalos definidos pelas coleções de restrições, o que compromete a escalabilidade do algoritmo e, consequentemente, a sua utilização em problemas práticos. Um exemplo destes últimos é a utilização do problema U-curve para realizar a estimação de redes gênicas regulatórias~\cite{u-curve algorithm}, uma aplicação de imediato interesse prático do \href{http://cetics.butantan.gov.br/en/platforms/computational-biology}{Grupo de Biologia Computacional e de Bioinformática} do CEPID \href{http://cetics.butantan.gov.br/en}{CeTICS} ({\em Center of Toxins, Immune-response and Cell Signaling.}).

\newpage

\section{Objetivos}

Neste trabalho, propomos atingir as seguintes metas:

\begin{enumerate}

  \item Investigar o uso de uma arquitetura que permita representar o espaço de busca corrente do algoritmo {\tt UCS} utilizando uma estrutura de dados mais eficiente do que listas ligadas encadeadas; neste caso, estudaremos o uso de diagramas de decisão binária reduzidos e ordenados ({\em Reduced Ordered Binary Decision Diagrams} -- ROBDDs)~\cite{bryant}. 
  
  \item Implementar a estrutura de dados baseada em ROBDDs. Propomos desenvolver uma versão modificada do algoritmo {\tt UCS} que utilize a nova estrutura de dados ao invés de listas duplamente encadeadas e realizar testes com instâncias artificiais e também com dados de aplicações reais.
  

\end{enumerate}

Na figura~\ref{fig:ROBDDs} mostramos um exemplo de propriedades interessantes que surgem quando escolhemos ROBDDs como estrutura de dados para realizar o controle do espaço de busca corrente do algoritmo {\tt UCS}; como podemos verificar na figura~\ref{fig:ROBDDs:A}, consultar se um elemento pertence ao espaço de busca corrente utilizando ROBDDs exige um tempo linear em relação a $n$ (precisamos percorrer um caminho da raiz até uma das folhas, o que exige no máximo a visita a $n + 1$ nós); o consumo de tempo da mesma busca via listas duplamente encadeadas é uma função do produto de $n$ pelo tamanho das coleções de restrições. Ademais, a propriedade de redução dessa estrutura de dados pode acarretar em uma maior eficiência no uso de memória do computador, que é proporcional ao nível de redução do diagrama produzido, o que por sua vez depende da ordem dos elementos de $S$ durante a construção da árvore de decisão binária (figuras~\ref{fig:ROBDDs:B} e \ref{fig:ROBDDs:C}).

Atingindo as metas estabelecidas, esperamos obter uma versão modificada do algoritmo {\tt UCS} que também seja eficiente do ponto de vista do consumo de tempo computacional, o que potencialmente resolverá o problema de escalabilidade da atual versão do algoritmo, possibilitando assim o seu uso na resolução de problemas práticos em Reconhecimento de Padrões, Morfologia Matemática e Biologia Computacional.

\begin{figure}[!ht]
  \centering 
    \subfigure[] {\scalebox{0.75}{\includegraphics[trim=2cm 17.5cm 2cm 2cm, clip=true]{Binary_tree_3.pdf}}  \label{fig:ROBDDs:A}}  
  \begin{tabular}{c c}
    \subfigure[] {\scalebox{0.75}{\includegraphics[trim=0cm 17.5cm 12.2cm 2cm, clip=true]{ROBDD_good.pdf}}  \label{fig:ROBDDs:B}}  
  &
        \subfigure[] {\scalebox{0.75}{\includegraphics[trim=1cm 15.5cm 12cm 2cm, clip=true]{ROBDD_bad.pdf}}  \label{fig:ROBDDs:C}}  
  \end{tabular}
  \caption{exemplo do uso de ROBDDs para representar o espaço de busca corrente durante a execução do algoritmo {\tt UCS}. Figura~\ref{fig:ROBDDs:A}: árvore binária que representa o espaço de busca definido pela coleção de restrições inferiores do exemplo mostrado na figura~\ref{fig:lower_restriction}; um arco sólido (pontilhado) representa que o elemento da ponta superior está (não está) presente no subconjunto considerado. Uma folha tem valor $1$ se um subconjunto está fora do espaço de busca corrente e $0$ caso contrário. Figura~\ref{fig:ROBDDs:B}: ROBDD obtido através da redução da árvore de decisão binária da figura~\ref{fig:ROBDDs:A}. Figura~\ref{fig:ROBDDs:C}: ROBDD obtido através da redução de uma árvore de decisão binária cuja raiz é o elemento $c$ ao invés de $a$; a redução dessa árvore produz um ROBDD com mais elementos do que o exibido na figura~\ref{fig:ROBDDs:B}.} 
  \label{fig:ROBDDs} 
\end{figure}



\section{Plano de trabalho}

A pesquisa se iniciará com o estudo do problema U-curve, do algoritmo {\tt UCS} e do arcabouço featsel. Em seguida, estudaremos ROBDDs. Nestes estudos iniciais, consideraremos como ponto relevante a investigação de métodos para atualizar um ROBDD de forma incremental, isto é, como atualizar a estrutura de maneira eficiente nos casos em que queremos mudar o valor correspondente a um único elemento do domínio da função Booleana que está implícita no ROBDD. A atualização incremental da estrutura é necessária, pois elementos do espaço de busca corrente precisam ser removidos do mesmo conforme são visitados pelo algoritmo {\tt UCS}. Uma possibilidade a ser investigada é o uso de operações Booleanas sobre árvores de decisão binária para realizar a atualização de ROBDDs.

Na sequência, será implementada, no arcabouço featsel, uma arquitetura de ROBDDs capaz de representar de maneira eficiente o espaço de busca corrente do algoritmo {\tt UCS}. Com a arquitetura pronta, implementaremos uma nova versão do {\tt UCS} com o uso de ROBDDs. Inicialmente testaremos o novo algoritmo ({\tt UCS-ROBDD}) com instâncias artificiais.

Os resultados dos primeiros testes serão avaliados. Essa avaliação, juntamente com estudos de métodos eficientes para construção de ROBDDs~\cite{rice,brace} (i.e., métodos que busquem minimizar o tamanho do diagrama obtido -- veja figura~\ref{fig:ROBDDs}), será empregada para construção e teste de uma nova versão do {\tt UCS-ROBDD}. Espera-se que este último minimize não somente o consumo de tempo computacional, como também o de memória. Vale observar que o problema de minimizar o tamanho de um ROBDD que descreve uma função Booleana é NP-difícil~\cite{rice}, o que implicará na provável utilização de heurísticas. Dessa forma, avaliaremos o impacto de diferentes métodos de minimização de diagramas na eficiência das variações do algoritmo {\tt UCS-ROBDD}.


Na tabela~\ref{tab:cronograma} exibimos um cronograma, no qual propomos uma lista de atividades para abordar o projeto e seus respectivos prazos. As descrições dessas atividades são apresentadas na seção~\ref{sec:atividades}


\subsection{Cronograma}

  \begin{table}[!ht]
    \caption{cronograma de atividades previstas neste projeto de Iniciação Científica. As descrições das atividades listadas seguem na seção a seguir.} \label{tab:cronograma}
    \begin{center}
      \smallskip
      \begin{tabular}{c ccc ccc}
        \toprule
        \small Atividade/mês & \small Dez.14 & \small Jan.15 & \small Fev.15 & \small Mar.15 &  \small Abr.15 & \small Mai.15\\ \hline

        \small Atividade 1   & \small {\bf x} & \small {\bf x} & \small - & \small - & \small - \\

        \small Atividade 2   & \small - & \small {\bf x} & \small {\bf x} & \small - & \small - \\
        
        \small Atividade 3   & \small - & \small {\bf x} & \small {\bf x} & \small {\bf x} & \small - &  \small - \\
        
        \small Atividade 4   & \small - & \small - & \small {\bf x} & \small {\bf x} & \small {\bf x} &  \small - \\
        
        \small Atividade 5   & \small - & \small - & \small - & \small {\bf x} & \small {\bf x} &  \small {\bf x} \\
        
        \small Primeiro Relatório   & \small - & \small - & \small - & \small - & \small {\bf x} &  \small {\bf x} \\
        \bottomrule
        \toprule
\small Atividade/mês & \small Jun.15 & \small Jul.15 & \small Ago.15 & \small Set.15 &  \small Out.15 & \small Nov.15\\ \hline
        \small Atividade 6   & \small {\bf x} & \small {\bf x} & \small {\bf x} & \small - & \small - &  \small -\\
        
        \small Atividade 7  & \small - & \small {\bf x} & \small {\bf x} & {\bf x} & \small - &  \small -\\
        
        \small Atividade 8  & \small - & \small - & \small {\bf x} & \small {\bf x} & \small {\bf x} &  \small {\bf x}\\
        
        \small Segundo Relatório   &  \small - & \small - & \small - & \small - & \small {\bf x} &  \small {\bf x}\\
        \bottomrule
      \end{tabular}
    \end{center}
  \end{table}

\subsection{Descrição das atividades} \label{sec:atividades}

\begin{itemize}

  \item[] {\bf Atividade 1.} Estudos do problema U-curve, do algoritmo UCS e do arcabouço featsel.

  \item[] {\bf Atividade 2.} Estudo de diagramas de decisão binária ordenados e reduzidos (ROBDDs); em particular, como atualizar um ROBDD de forma incremental.

  \item[] {\bf Atividade 3.} Estudo de atualização de ROBDDs e desenvolvimento de uma arquitetura de ROBDDs para representar o espaço de busca corrente no algoritmo {\tt UCS}.

  \item[] {\bf Atividade 4.} Implementação da nova arquitetura dentro do arcabouço featsel, construindo uma versão inicial do algoritmo {\tt UCS-ROBDD}.

  \item[] {\bf Atividade 5.} Testes da primeira versão do algoritmo com instâncias artificiais.

  \item[] {\bf Atividade 6.} Aperfeiçoamento da arquitetura e do algoritmo de acordo com os resultados dos testes, implicando numa segunda versão do algoritmo. Investigaremos métodos eficientes de ordenação de variáveis para construção de ROBDDs, e como utilizar tais métodos nas atualizações incrementais da arquitetura de ROBDDs desenvolvida nas etapas anteriores do projeto.

  \item[] {\bf Atividade 7.} Testes da segunda versão do algoritmo, com instâncias artificiais.

  \item[] {\bf Atividade 8.} Estudos comparativos entre a segunda versão do algoritmo e outros algoritmos já implementados no arcabouço featsel com testes de instâncias artificiais e práticas.

\end{itemize}

\section{Materiais e métodos}

Para implementar e testar o uso da nova estrutura de dados para representar coleções de restrições inferiores e superiores, baseada em ROBDDs, utilizaremos o arcabouço featsel. Esse arcabouço foi desenvolvido seguindo o paradigma de Orientação a Objetos na linguagem C++ e possui classes que facilitam a criação de novos algoritmos de seleção de características que organizam o espaço de busca como um reticulado Booleano~\cite{msreis thesis}. 

Outro fator que facilita a utilização do arcabouço featsel é que ele possui licença \textit{GNU General Public License} (GNU-GPL), isto é, seu código é aberto e livre. Portanto, podemos modificar o código do arcabouço conforme for necessário para o desenvolvimento de novos algoritmos e/ou para realização de todos os testes necessários para avaliar o desempenho dos mesmos.


\section{Formas de análise dos resultados}

Avaliações dos algoritmos que serão desenvolvidos utilizarão como métricas principais o consumo de memória e o tempo computacional necessário para uma execução. Faremos essas medições utilizando o arcabouço featsel, uma vez que esse ambiente já proporciona métodos para registrar o consumo de tempo de um algoritmo e também de suas subrotinas, assim como a memória que é exigida para armazenar as coleções de restrições ao longo de uma execução do mesmo.

Para realizar os testes, empregaremos como entradas instâncias artificiais ``difíceis" do problema U-curve (e.g., obtidas através de redução polinomial do problema da soma de subconjuntos), além de instâncias de problemas reais tais como projeto W-operadores e estimação de redes gênicas regulatórias.


\begin{thebibliography}{9}

\addcontentsline{toc}{section}{Referências}

\bibitem{u-curve algorithm}
Ris, Marcelo, Junior Barrera, and David C. Martins Jr. \emph{U-curve: A branch-and-bound optimization algorithm for U-shaped cost functions on Boolean lattices applied to the feature selection problem}. Pattern Recognition, 43.3 (2010): 557-568.

\bibitem{msreis thesis}
Reis, Marcelo S. "Minimization of decomposable in U-shaped curves functions defined on poset chains–algorithms and applications." PhD thesis, Institute of Mathematics and Statistics, University of São Paulo, Brazil, (2012).

\bibitem{iubb}
Atashpaz-Gargari, Esmaeil, Ulisses M. Braga-Neto, and Edward R. Dougherty. "Improved branch-and-bound algorithm for U-curve optimization." 2013 IEEE International Workshop on Genomic Signal Processing and Statistics (GENSIPS), (2013).

\bibitem{ucs paper}
Reis, Marcelo S., Carlos E. Ferreira, and Junior Barrera. "The U-curve optimization problem: improvements on the original algorithm and time complexity analysis." arXiv preprint arXiv:1407.6067, (2014). 

\bibitem{math morph paper}
Reis, Marcelo S., and Junior Barrera. "Solving Problems in Mathematical Morphology through Reductions to the U-Curve Problem." Mathematical Morphology and Its Applications to Signal and Image Processing. Springer Berlin Heidelberg, (2013): 49-60.

\bibitem{bryant}
Bryant, Randal E. "Graph-based algorithms for boolean function manipulation." IEEE Transactions on Computers, 100.8 (1986): 677-691. 

\bibitem{rice}
Rice, Michael and Sanjay Kulhari. "A survey of static variable ordering heuristics for efficient BDD/MDD construction." University of California, Tech. Rep., (2008)

\bibitem{brace}
Brace, Karl S., Richard L. Rudell, and Randal E. Bryant. "Efficient implementation of a BDD package." Proceedings of the 27th ACM/IEEE design automation conference, (1991). 


\end{thebibliography}

\end{document}



