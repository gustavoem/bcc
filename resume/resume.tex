%!TEX TS-program = xelatex
%!TEX encoding = UTF-8 Unicode
% Awesome CV LaTeX Template for CV/Resume
%
% This template has been downloaded from:
% https://github.com/posquit0/Awesome-CV
%
% Author:
% Claud D. Park <posquit0.bj@gmail.com>
% http://www.posquit0.com
%
% Template license:
% CC BY-SA 4.0 (https://creativecommons.org/licenses/by-sa/4.0/)
%

%-------------------------------------------------------------------------------
% CONFIGURATIONS
%-------------------------------------------------------------------------------
% A4 paper size by default, use 'letterpaper' for US letter
\documentclass[11pt, a4paper]{awesome-cv-res}

% Configure page margins with geometry
\geometry{left=1.4cm, top=.8cm, right=1.4cm, bottom=1.8cm, footskip=.5cm}

% Specify the location of the included fonts
\fontdir[./fonts/]

% Color for highlights
% Awesome Colors: awesome-emerald, awesome-skyblue, awesome-red, awesome-pink, awesome-orange
%                 awesome-nephritis, awesome-concrete, awesome-darknight
\colorlet{awesome}{awesome-red}
% Uncomment if you would like to specify your own color
 \definecolor{awesome}{HTML}{14255F}

% Colors for text
% Uncomment if you would like to specify your own color
\definecolor{darktext}{HTML}{414141}
% \definecolor{text}{HTML}{333333}
\definecolor{graytext}{HTML}{5D5D5D}
\definecolor{lighttext}{HTML}{999999}

% Set false if you don't want to highlight section with awesome color
\setbool{acvSectionColorHighlight}{false}

% If you would like to change the social information separator from a pipe (|) to something else
\renewcommand{\acvHeaderSocialSep}{\quad\textbar\quad}


%-------------------------------------------------------------------------------
%	PERSONAL INFORMATION
%	Comment any of the lines below if they are not required
%-------------------------------------------------------------------------------
\name{Gustavo Estrela de Matos}{}
%\address{235 Agostinho dos Santos Street, São Paulo - São Paulo}
\position{Tech Lead @ Geekie}
\mobile{(+55) 011 975-934-129}
\email{estrela.gustavo.matos@gmail.com}
\github{gustavoem}
\idioms{\textbf{Native} Portuguese | \textbf{Fluent} English}


%-------------------------------------------------------------------------------
\begin{document}

% Print the header with above personal informations
\makecvheader
% Print the footer with 3 arguments(<left>, <center>, <right>)
% Leave any of these blank if they are not needed
\makecvfooter
{}
{}
{\thepage}

%-------------------------------------------------------------------------------
%     EXPERIENCES
%-------------------------------------------------------------------------------
\cvsection{Experiences}
\begin{cventries}
%---------------------------------------------------------
\cventry
{Tech Lead}
{Geekie}
{São Paulo, Brazil}
{May 2022 - Present}
{
    Led a team of 4 to 7 developers including me (for most of the time, less
    than 5 developers). Worked with React/React-Native front-end and
    Python back-end with Flask/Pyramid frameworkds.
    My responsabilities were:
    \newline \hspace{1em} • working with our Product Manager and 
    Designer, giving visibility of possible solutions and their costs 
    and risks;
    \newline \hspace{1em} • guaranteeing the focus of the team on the sprint goal, trying to
    block possible interferences;
    \newline \hspace{1em} • supporting team developers, on their technical progression.
    \newline I also worked in some projects outside the scope of my squad. 
    As an instance, for the engineering team, I migrated legacy 
    back-ends and packages from Python 2 to Python 3; I was also 
    part of a group that started and experimented using GraphQL in new 
    features. 
    I also helped other Geekie areas, such as Marketing and Assessment 
    Tests Creation (team that created exams in many formats, including
    ENEM) producing data extractions that were useful for their teams, 
    usually with Python Notebooks or BigQuery tables.
}    
    
\cventry
{Software Engineer}
{}
{}
{January 2020 - May 2022}
{Worked as a full stack engineer in squads of 4 to 6 people. I worked on
the development of new features for the Geekie One app, including tasks 
of the back-end, in Python, and tasks of the front-end, where we used 
code-sharing with React and React-Native. Designed a solution that 
allowed our application to provide, in a few seconds, reports that
aggregated performance of many students, with different granularity
\newline\textbf{Skills:}  Redis Queue, Iron.io, Heroku, AWS EC2, AWS
    Lambda, Flask, React.js, React Native, Python, MongoDB, Google 
    BigQuery, Relational Databases, Object-Relational Mapping (ORM).
}

\cventry
{Graduate Researcher (with FAPESP Scholarship)}
{Butantan Institute}
{São Paulo, Brazil}
{January 2018 - February 2021}
{\href{https://bv.fapesp.br/en/bolsas/175684/identification-of-cell-signaling-pathways-based-on-biochemical-reaction-kinetics-repositories/}
{\color{awesome}\underline {Masters project}} in a team of 2 (student 
    and advisor). We created a Python program 
    (github.com/gustavoem/SigNetMS) that allowed Cell Signaling Model
    Selection, using Bayesian methods to create estimates of marginal
    likelihood of models.
}

\cventry
{Undergraduate Researcher (with FAPESP Scholarship)}
{}
{}
{May 2017 - December 2017}
{\href{https://bv.fapesp.br/en/bolsas/170553/design-of-poset-forest-based-algorithms-for-the-u-curve-optimization-problem/}
{\color{awesome} \underline{Scientific initiation}} in a team of 2 
    (student and advisor). We created new parallel 
    algorithms for the U-Curve problem, using graph partition.  
    Results were presented as a
    \href{https://gustavoem.github.io/ucurve-pfs/index.html}{\color{awesome}
    \underline{conclusion project}} for the bachelor title in computer
    science and also on a
    \href{https://www.mdpi.com/1099-4300/22/4/492}{\color{awesome}\underline{published paper}}.}

\cventry
{Undergraduate Researcher (with FAPESP Scholarship)}
{}
{}
{January 2015 - July 2015}
{\href{https://bv.fapesp.br/en/bolsas/156441/studies-of-efficient-data-structures-to-tackle-the-u-curve-optimization-problem/}
{\color{awesome} \underline{Scientific initiation}} in a team of 2 
    (student and advisor). We investigated the usage of new  data 
    structures for the U-Curve problem. We created a  new algorithm, UCSR, implemented in C++ and described on a 
    \href{https://www.sciencedirect.com/science/article/pii/S0020025518306789?via\%3Dihub}
    {\color{awesome}\underline{published paper}}.
\newline\textbf{Skills:} Scientific Research, SciPy, NumPy, SymPy,
celery, pandas, Machine Learning, Bayesian Inference, Graph Theory,
Optimization, Parallel Computing, C++, OpenMP. 
}
\end{cventries}
%---------------------------------------------------------



%-------------------------------------------------------------------------------
%     EDUCATION
%-------------------------------------------------------------------------------
\cvsection{Education}
\begin{cventries}
%---------------------------------------------------------
\cventry
{Master of Science in Computer Science}
{Institute of Mathematics and Statistics (University of São Paulo)} % Institution
{São Paulo, Brazil} % Location
{January 2018 - February 2021}
{Dissertation of title "Identification of cell signaling pathways 
based on biochemical reaction kinetics repositories". Research project 
awarded with a São Paulo Research Foundation (FAPESP) scholarship.}
\newline

\cventry
{Bachelor of Science in Computer Science} % Degree
{} % Institution
{} % Location
{February 2013 - December 2017} % Date(s)
{Ranked \#3 out of 50 students. GPA: 9/10.}
\newline

\cventry
{Science Without Borders, Study abroad program in Computer Science}
{Texas A\&M University}
{College Station, Texas}
{September 2015 - May 2016}
{}
\end{cventries}
%---------------------------------------------------------

%-------------------------------------------------------------------------------
%     PUBLICATIONS
%-------------------------------------------------------------------------------
%\cvsection{Publications}
%\begin{cventries}
%\cventry
    %{REIS, MARCELO S.; ESTRELA, GUSTAVO; FERREIRA, CARLOS EDUARDO; BARRERA, JUNIOR.}
    %{featsel: A framework for benchmarking of feature selection algorithms and cost functions.}
    %{SoftwareX, v. 6, p. 193-197, 2017.}
    %{}
    %{In the context of machine learning, feature selection is an 
    %interesting tool to reduce data complexity. Since the feature 
    %selection problem is NP-hard, many algorithms exist to solve the 
    %problem and a standard benchmark is desirable. On this paper we 
    %describe featsel, a framework implemented in C++ to construct and 
    %benchmark different feature selection algorithms and cost 
    %functions.}
%\cventry
    %{REIS, MARCELO S.; ESTRELA, GUSTAVO; FERREIRA, CARLOS EDUARDO; BARRERA, JUNIOR.}
    %{Optimal Boolean lattice-based algorithms for the U-curve optimization problem.}
    %{Information Sciences, \newline2018.}
    %{}
    %{The U-curve problem is a particular case of the feature selection
    %problem when the cost function describes U-shaped curves in every
    %chain of search space. In this paper we present two new algorithms
    %and compare them to other state of the art algorithms using the 
    %featsel framework.}
%\end{cventries}
%---------------------------------------------------------

%-------------------------------------------------------------------------------
%     Certifications and Awards
%-------------------------------------------------------------------------------
\vspace{-1.2em}
\cvsection{Certifications and Awards}

\descriptionstyle{\textbf{High Academic Merit Award} awarded by the
Computer Science Department of the Institute of Mathematics and
Statistics of the University of São Paulo. April 2018.}

\descriptionstyle{\textbf{TOEFL iBT} total score 93/120.}
%---------------------------------------------------------


%\cvskill
%\cvskill
%{Portuguese} % Category
%{Native}


%-------------------------------------------------------------------------------
%     EXTRA-CURRICULAR ACTIVITIES
%-------------------------------------------------------------------------------
%\cvsection{Other}
%\begin{cventries}
%\cventry
    %{Member\vspace{-3em}}
    %{\vspace{-3em}Texas A\&M Table Tennis Club}
%{}
%{January 2016 - May 2016}
%{}

%\cventry
%{Member and Co-founder}
%{University of São Paulo Open Source Hardware Student Group}
%{}
%{November 2013 - August 2015}
%{The group was founded by computer science students in an attempt to 
%foment the studies of open source hardware such as Arduino between 
%students and local community. }

%\cventry
%{Member}
%{Computer Science Class' Representative}
%{}
%{February 2013 - August 2015}
%{Class representatives in our department are channels of communication
    %between students and professors. I also participated on the 
    %department project of gathering lecture quality feedback from 
    %students.}

%\cventry
%{Member and organizer}
%{Institute of Mathematics and Statistics Baseball and Softball Club}
%{}
%{February 2013 - Present}
%{I participate in many practice sessions and championships of the club 
    %every year. During the period of 2014 I worked as the club 
    %treasurer.}

%---------------------------------------------------------
%\end{cventries}
%---------------------------------------------------------
\end{document}
