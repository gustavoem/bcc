\documentclass[letterpaper,11pt]{article}
%\documentstyle[11pt]{article}

\usepackage{hyperref} %% THIS SEEMS TO FIX PDFTex' pagesize problem!
\usepackage{mytex}
\usepackage{fullpage}
\usepackage{headerfooter}
\usepackage{paralist}

\pageheader{CPSC--410/611/613}{}{Machine Problem 3}
\pagefooter{{\tiny Ver. 2012C}}{}{Page \thepage}

\begin{document}
\begin{center}
{\large {\bf Machine Problem 3: Virtual Memory Management and Memory Allocation}\\[0.3in]}
\end{center}

\subsection*{Introduction}

In this machine problem we continue our investigation of demand-paging 
based virtual memory system. For this we extend our solution for MP2 in two directions: 
\begin{itemize}
\item First, we extend the page table management to support very large numbers and sizes of
  address spaces. For this, we must place the page table into {\em
    virtual memory}. As you will see, this will slightly complicate
  the page table management and the design of the page fault
  handlers. 
\item Second, we will implement a simple memory allocator and hook it
  up to the {\tt new} and {\tt delete} operators of C++.
\end{itemize}
As a result we will have a pretty flexible simple memory management system
that we will be able to use for later machine problems. In particular,
we will be able to dynamically allocate memory in a fashion that is
familiar to us from standard user-level C++ programming.

\subsection*{Support for Large Address Spaces}

To implement the page table management in the previous machine problem
it was sufficient to store the page table directory page and any page
table pages in directly-mapped frames. Since the logical address of
these frames is identical to their physical address, it was very
simple to manipulate the content of the page table directory and of
the page table frames. This approach works fine when the number of
address spaces and the size of the requested memory is small;
otherwise, we very quickly run out of frames in the directly-mapped
frame pool.

In the current machine problem, we circumvent the size limitations of
the directly-mapped memory by allocating page table pages (and
possibly even page table directories if you want) in mapped memory,
i.e. memory above 4MB in our case.

\subsubsection*{Help! My Page Table Pages are in Mapped Memory!}
It will be pretty straightforward to move your page-table
implementation from direct-mapped memory to mapped memory. 
When paging is turned on, the CPU issues logical
addresses, and you will have problems working with the page table when
you place it in mapped memory. In particular, you will want to modify
entries in the page directory and page table pages, but the CPU can
only issue logical addresses. 
You can maintain the mapping from logical addresses of page table
pages and page directory to physical addresses by setting up
complicated tables to do so. Fortunately, you have the page table that
does this already for you. You simply need to find a way to make use
of it. 
Tim Robinson's tutorial ``Memory Management 1''
(\url{http://www.osdever.net/tutorials/view/memory-management-1})
briefly addresses this problem. We will use
the trick described by Robinson to have the last entry in the page
table directory point back to the beginning of the page table.

\subsubsection*{Recursive Page Table Look-up}

Both the page table directory and the page table pages contain
physical addresses. If a logical address of the form 
\begin{center}
{\tt | X : 10 | Y : 10 | offset : 12 |}\footnote{This expression represents
  a 32-bit value, with the first 10 bits having value {\tt X}, the following
  10 bits having value {\tt Y}, and the last 12 bits having value {\tt offset}.}
\end{center}
is issued by the CPU, the memory management unit (MMU) will use the
first 10 bits (value {\tt X}) to index into the page directory (i.e.,
relative to the Page Directory Base Register) to look up the Page
Directory Entry (PDE). The PDE points to the appropriate page table
page. The MMU will use the second 10 bits (value {\tt Y}) of the address to
index into the page table page pointed to by the PDE to get the Page
Table Entry (PTE). This entry will contain a pointer to the physical
frame that contains the page.

If we set the last entry in the page directory to point to the
page directory itself, we can play a number of interesting tricks. For
example, the address 
\begin{center}
{\tt | 1023 : 10 | 1023 : 10 | offset : 12 |}
\end{center}
will be resolved by the MMU as follows: The MMU will use the first 10
bits (value 1023) to index into the page directory to look up the
PDE. PDE number 1023 points to the page directory itself. The MMU does
not know about this and treats the page directory like any other page
table page: It uses the second 10 bits to index into the (supposed)
page table page to look up the PTE. Since the second 10 bits of the
address are have value 1023, the resulting PTE points again to the
page directory itself. Again, the MMU does not know about this and
treats the page directory like any frame : It uses the offset to index
into the physical frame. This means that the offset is an index to a
byte in the page directory. If the last two bits of the offset are
zero, the offset becomes an index to (offset DIV 4)'th entry in the
page directory. In this way you can manipulate the page directory if
it is in logical memory. Neat!

Similarly, the address 
\begin{center}
{\tt | 1023 : 10 | X : 10 | Y : 10 | 0 : 2 |}
\end{center}
gets processed by the MMU as follows: The MMU will use the first 10
bits (value 1023) to index into the page directory to look up the
PDE. PDE number 1023 points to the page directory itself. The MMU does
not know about this and treats the page directory like any other page
table page: It uses the second 10 bits (value {\tt X}) to index into
the (supposed) page table page to look up the PTE (which in reality is
the {\tt X}th PDE). The offset is now used to index into the
{supposed) physical frame, which is in reality the page table page
  associated with the {\tt X}th directory entry. Therefore, the
  remaining 12 bits can be used to index into the {\tt Y}th entry in
  the page table page.

The two examples above illustrate how one can manipulate a page
directory that is stored in virtual memory (i.e. not stored in
directly-mapped memory in our case) or a page table that is stored in
virtual memory, respectively.

\subsection*{An Allocator for Virtual Memory}

In the second part of this machine problem we will design and
implement an {\bf allocator for virtual memory}. This allocator will
be realized in form of the following virtual-memory pool class {\tt VMPool}:
\begin{verbatim}
class VMPool { /* Virtual Memory Pool */
private:
     /* -- DEFINE YOUR VIRTUAL MEMORY POOL DATA STRUCTURE(s) HERE. */

public:
   VMPool(unsigned long _base_address,
          unsigned long _size,
          FramePool *   _frame_pool
          PageTable *   _page_table);
   /* Initializes the data structures needed for the management of this
      virtual-memory pool. 
      _base_address is the logical start address of the pool. 
      _size is the size of the pool in bytes.
      _frame_pool points to the frame pool that provides the virtual 
      memory pool with physical memory frames.
      _page_table points to the page table that maps the logical memory
      references to physical addresses. */

   unsigned long allocate(unsigned long _size);
   /* Allocates a region of _size bytes of memory from the virtual 
      memory pool. If successful, returns the virtual address of the
      start of the allocated region of memory. If fails, returns 0. */

   void release(unsigned long _start_address);
   /* Releases a region of previously allocated memory. The region
      is identified by its start address, which was returned when the
      region was allocated. */

   BOOLEAN is_legitimate(unsigned long _address);
   /* Returns FALSE if the address is not valid. An address is not valid
      if it is not part of a region that is currently allocated. */
};
\end{verbatim}

An address space can have {\bf multiple virtual memory pools} (created by
constructing multiple objects of class {\tt VMPool}), with each pool
having {\bf multiple regions}, which are created by the function {\tt
  allocate} and destroyed by the function {\tt release}. 
Our virtual-memory pool will be a somewhat lazy allocator: Instead of
immediately allocating frames for a newly allocated memory region,
the pool will simply ``remember'' that the region exists by storing
start address and size in a local table. Only when a reference to a
memory location inside the region is made, and a page fault occurs
because no frame has been allocated yet, the page table (this is a
separate object) finally allocates a frame and makes the page valid.
In order for the page table object to know about virtual memory pools,
we have the pools {\bf register} with the page table by calling a
function {\tt PageTable::register(VMPool * \_pool)}. The page table
object maintains a list (or likely an array) of references to virtual
memory pools. 
When a virtual memory region is deallocated (as part of a call to {\tt
  VMPool::release()}, the virtual memory pool informs the page table
that any frames allocated to pages within the region can be freed and that
the pages are to be invalidated. In order to simplify the
implementation, we have the virtual memory pool call the function {\tt
  PageTable::free\_page(unsigned int page\_no)} for each page that is to
be freed.

\subsubsection*{Implementation Issues:}

There are no limits to how much you can optimize the implementation of
your allocator. Keep the following points in mind when you design your
virtual memory pool in order to keep the implementation simple. 
\begin{itemize}
\item Ignore the fact that the function {\tt allocate} allows for the
  allocation of arbitrary-sized regions. Instead, always allocate
  multiples of pages. In this way you won't have to deal with
  fractions of pages. Except for some internal fragmentation, the user
  would not know the difference.
\item Don't try to optimize the way how frames are returned to the
  frame pool. Whenever a virtual memory pool releases a region, notify
  the page table that the pages can be released (and any allocated
  frames can be freed).
\item Keep the implementation of the allocator simple. There is no
  need to implement a Buddy-System allocator, for example. A simple
  list of allocated regions, or something similar, should suffice.
\item Even maintaining a list of allocated regions can be a bit of a
  challenge, given that we don't have a memory allocator yet. This is
  where the frame pool comes in handy. Use the frame pool to store an
  array of region descriptors. In the constructor of the virtual
  memory pool request a frame to store the region descriptor
  list. This solution limits the number of regions that you can
  allocate. You can easily eliminate this limitation by requesting a
  new overflow frame whenever the number of regions grows too big, and
  you run out of space.
\item A new virtual memory pool {\bf registers} with the page table
  object. In this way the page table can check whether memory
  references are legitimate (i.e., they are part of previously
  allocated regions). Whenever the page table experiences a page
  fault, it checks with the registered virtual memory pools whether
  the address is legitimate by calling the function {\tt
    VMPool::is\_legitimate} on the registered pools. If it is, 
  the page table proceeds to
  allocate a frame; otherwise, the memory reference is invalid.
\item At this time we don't have a backing store yet, and pages
  cannot be ``paged out''. This means that we can easily run out of
  memory if a program allocates multiple regions and then references
  lots of pages in the allocated regions. Don' worry about this for
  now. We will add page-level swapping in a later MP.
\end{itemize}

\subsubsection*{Modifications to Class {\tt PageTable}}

In this machine problem you will modify the class {\tt PageTable} in the
following ways:
\begin{enumerate}
\item Add support for page tables in virtual memory. This has been
  described above.
\item Add support for registration of virtual memory pools. In
  order to do this, the following function is to be provided:
  \begin{center}
  {\tt void PageTable::register(VMPool * \_pool);}
  \end{center}
  The page table object maintains a list of registered pools.
\item Add support for virtual memory pools to request the release of
  previously allocated pages. The following function is to be
  provided:
  \begin{center}
  {\tt void PageTable::free\_page(unsigned long \_page\_no);}
  \end{center}
  If the page is valid, the page table releases the frame and marks
  the page invalid.
\item Add support for region check in page fault handler. Whenever a
  page fault happens, check with registered pools to see whether the
  address is legitimate. If it is, proceed with the page
  fault. Otherwise, abort.
\end{enumerate}

\subsection*{The Assignment}

\begin{enumerate}
\item Read  Tim Robinson's tutorial ``Memory
  Management 1''
  (\url{http://www.osdever.net/tutorials/view/memory-management-1}) to
  understand some of the intricacies of setting up a memory manager.
\item Extend your page table manager from MP2 to handle pages in
  virtual memory. Use the ``recursive page table lookup'' scheme
  described in this handout.
\item Extend your page table manager to (1) handle registration of virtual
  memory pools, (2) handle requests to free pages, and (3) check for
  legitimacy of logical addresses during page fault.
\item Implement a simple virtual memory pool manager as defined in
  file {\tt vm\_pool.H}. Always allocate multiples of pages at a
  time. This will simplify your implementation.
\end{enumerate}

You should have access to a set of source files, BOCHS environment
files, and a makefile that should make your implementation easier. In
particular, the {\tt kernel.C} file will contain documentation that
describes where to add code and how to proceed about testing the code
as you progress through the machine problem. The updated interface for the page table is 
available in {\tt page\_table.H} and the interface for the virtual memory pool manager
is available in file {\tt vm\_pool.H}.

\subsection*{What to Hand In}
You are to hand in the following items on CSNET:
\begin{itemize}
\item A ZIP file containing the following files:
  \begin{enumerate}
  \item A design document (in PDF format) that describes your
    implementation of the page table and the virtual memory pool.
  \item A pair of files, called {\tt page\_table.H} and {\tt
      page\_table.C}, which contain the definition and implementation
    of the required functions to initialize and enable paging, to
    construct the page table, to handle registration of virtual memory
    pools, and to handle page faults.  Any modification to the
    provided {\tt .H} file must be well motivated and documented.
    %% rb 11-10-25
    %% The following was added at the request of TA Suneil Mohan. 
    %% He was having problems with grading because some students 
    %% were relying on their own implementations of frame pools.
  \item A pair of files, called {\tt frame\_pool.H} and {\tt
      frame\_pool.C}, which contain the definition and implementation
    of the frame pool. (These two files are for the benefit of the
    grader only. If you have not modified these files, simply submit
    the ones that were handed out to you. Some students have modified
    them in the previous machine problem, which has made it difficult for the TA
    to grade the submissions.)
  \item A pair of files, called {\tt vm\_pool.H} and {\tt vm\_pool.C},
    which contain the definition and implementation of the virtual
    memory pool. Any modifications to the provided file {\tt
      vm\_pool.H} must be well motivated and documented.
  \end{enumerate}

%% rb 11-10-25
%% This was added at the request of TA Suneil Mohan, who was having problems
%% with file names.
\item {\bf Note:} Pay attention to the capitalization in file names. For
  example, if we request a file called {\tt file.H}, we want the file
  name to end with a capital {\tt H}, not a lower-case one. While
  Windows does not care about capitalization in file names, other
  operating systems do. This then causes all kinds of problems when the 
  TA grades the submission.
  
\item Grading of these MPs is a very tedious chore.  These handin
  instructions are meant to mitigate the difficulty of grading, and to
  ensure that the grader does not overlook any of your efforts.

\item {\bf Failure to follow the handin instructions will result in
    lost points.}

\end{itemize}

\end{document}

